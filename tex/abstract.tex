\chapter*{Abstract}
\addcontentsline{toc}{chapter}{Abstract}

When people have to perform data entry tasks such as entering expenses, this involves retrieving data from one or more sources in the environment and entering it into a computer system. Switching between entering data and looking up the required data can be disruptive, as people have to pay a time cost each time they resume their data entry task. Furthermore, the cost to access these sources may vary, which can further influence the disruptiveness of looking up the information. In order to design interfaces that support people in their data entry work, it is therefore important to understand how these differences in access affect switching behaviour between entering and retrieving data. 

This thesis investigates how information access costs (IAC) affect how people manage subtasks of looking up information for a data entry task. Six studies are reported across three chapters, to understand the impact of IAC in the context of entering expenses in a finance office setting.

The first part of the thesis reports a series of qualitative studies that look at the context in which office workers in financial offices perform data entry tasks.  For entering expenses, people often had to go out of the data entry system to look up information. A planned study aims to further investigate the information sources needed for this task and their IAC, and the strategies people currently use to retrieve information from these sources.

The second part of the thesis reports a series of quantitative studies that investigate the extent to which observed behaviour of the first studies are due to the access costs of the information sources involved. When people copy from one source and IAC increases, people make fewer switches to look at this source, and instead enter what they have memorised. This makes people more efficient, but slightly increases errors. A planned study aims to further investigate switching behaviour when people have to retrieve the information from multiple sources.

The final part of the thesis then reports two studies that explore if design interventions can make people adopt more desirable strategies, i.e. strategies that minimise switching between entering data and looking up the required data. 

The primary contribution of this thesis is to show the fragmented nature of an expenses task, and that the IAC of required information sources affects how people manage how they look up certain information, which in turns impacts accuracy and efficiency. This finding has implications for the design of the current expenses system, which are validated through demonstration of how changing certain design features can potentially better support people in managing these subtasks of looking up information for a data entry task.