\chapter*{Abstract}
\addcontentsline{toc}{chapter}{Abstract}
Computer-based work often involves looking up information from different sources. Though these interruptions are required for work, switching away from a task can be disruptive: it slows people down, increases errors and it is challenging to remain focused on work. This thesis investigates how interruption management tools can better support people in managing these types of work-required interruptions in the context of data entry work.

The first part of the thesis describes two qualitative studies looking at the context in which office workers in finance offices perform data entry tasks. These studies reveal that participants often have to self-interrupt to collect information from both physical and digital sources. Whereas physical interruptions are postponed if these are expected to take time, digital interruptions were always addressed immediately as these were presumed to be quick. Observations however revealed that these could take much longer than expected, which suggests people are not aware of the time they spend on digital interruptions. 
The second part of the thesis reports three controlled experiments to test the hypothesis that people prioritise collecting information according to expected time costs. These studies show that, in a controlled setting where participants can learn the time costs involved in accessing information, they first switch to information sources that are fast to access, and switch more frequently to these sources. On the other hand, people either prepare or postpone looking up information which takes time.
The third part of the thesis evaluates whether giving people feedback on the duration of interruptions can influence people's switching strategies and data entry performance. An intervention was developed and evaluated with both an experimental data entry task and people's own data entry work. Together these studies show that feedback made people reflect on what they were doing during interruptions, it shortened the duration of interruptions and reduced errors and made people faster.

This thesis demonstrates that people manage interruptions based on expected time costs, and that giving people feedback on the time they spend on interruptions can help them manage their interruptions better. It makes a theoretical contribution by showing how people adapt to small changes in time costs not only by changing the number and duration of switches, but also the scheduling of these switches during the main task. It makes a practical contribution by showing how giving people feedback on time costs can help them in reducing the duration of interruptions, and improve people’s focus. 