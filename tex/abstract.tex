\chapter*{Abstract}
\addcontentsline{toc}{chapter}{Abstract}
Data entry work often involves retrieving data from one or more sources in the environment and entering it into a computer system. Though this type of self-interruption is required to complete the task, switching between looking up and entering the required data can be time-consuming and disruptive, and it can be difficult to remain focused on the task. Furthermore, the cost to access different information sources may vary, which can further influence the disruptiveness of these `information interruptions'. Though the phenomenon of work fragmentation and interruptions is well-researched, it is unclear when, how often, and how long people self-interrupt themselves for a data entry task, and whether the type of, and access to, information sources involved influences their strategies.

%When people have to perform data entry tasks such as entering expenses, this involves retrieving data from one or more sources in the environment and entering it into a computer system. Switching between entering data and looking up the required data can be disruptive, as people have to pay a time cost each time they resume their data entry task. Furthermore, the cost to access these sources may vary, which can further influence the disruptiveness of looking up the information. In order to design interfaces that support people in their data entry work, it is therefore important to understand how these differences in access affect switching behaviour between entering and retrieving data. 

This thesis investigates how information access costs (IAC) affect the number, duration, and timing of information interruptions for a data entry task. Seven studies are reported across three chapters to understand the impact of IAC in the context of entering expenses in a finance office setting. 

The first part of the thesis describes two qualitative studies looking at the context in which office workers in finance offices perform data entry tasks. Interview findings from Study 1 revealed that many data entry tasks have to be scheduled over time, and a critical component of data entry work is not just entering the data, but also retrieving data from multiple sources distributed in the environment. Participants explained that they batched similar tasks to efficiently complete their work, and held items in memory whilst switching between sources. Observations in Study 2 revealed that people adopt different strategies when organising information from physical or digital sources. Physical sources take time to access and participants therefore prepare it beforehand, or postpone retrieving it until a more convenient moment in the task. Digital sources are retrieved using the same device as that where the data entry occurs, and participants often interrupt their main entry task to switch between different windows and look up this information as soon as they need it. These switches can often take longer than intended, and participants were observed being logged out of the entry system, resuming the wrong data entry task, and reported it took time to resume their work after these longer switches.

%The first part of the thesis reports a series of qualitative studies that look at the context in which office workers in financial offices perform data entry tasks.  For entering expenses, people often had to go out of the data entry system to look up information. A planned study aims to further investigate the information sources needed for this task and their IAC, and the strategies people currently use to retrieve information from these sources.

The second part of the thesis reports three lab experiments that further test the influence of information access costs on people's information retrieval strategies. These studies show that, in a controlled setting where participants can learn the time costs involved in accessing information, they first switch to information sources that are fast to access, and switch more frequently to these sources. On the other hand, people either prepare or postpone looking up information which takes time. [Study 3 showed that if people retrieve all data from the same source, they will reduce switches between entering and looking up data if the access costs to this source increases. As it took more time to access, offloading behaviour was observed as well, and several participants placed items they were going to need nearby, but did not use them yet]. Study 4 further demonstrates that when people have to manage multiple sources, they collect and group items that are quick to access first, and leave items that take longer to access until the end. Study 5 shows that this effect also applies in a multi-task setup: when dealing with two data entry tasks, people will interleave between tasks in order to enter items with a low IAC first. As a result, participants made more omission errors and submitted tasks before they had completed entering all the items.  xx

%The second part of the thesis reports a series of quantitative studies that investigate the extent to which observed behaviour of the first studies are due to the access costs of the information sources involved. When people copy from one source and IAC increases, people make fewer switches to look at this source, and instead enter what they have memorised. This makes people more efficient, but slightly increases errors. A planned study aims to further investigate switching behaviour when people have to retrieve the information from multiple sources.

The final part of the thesis reports two studies that evaluate the effectiveness of a design intervention which aims to make information access cost more salient, and gives users explicit feedback on time spent to access information. Study 6 found that using an experimental data entry task, people who were shown how long they were away for made shorter switches, were faster to complete the task and made fewer data entry errors. Study 7 evaluated the intervention with finance workers processing expenses. Quantitative data replicated the findings from study 6 in-the-wild -  participants with the intervention made shorter interruptions during the period that interruptions were logged. Data from post-study interviews indicated that time feedback made participants more aware of their switches, and they tried to remain focused on looking up information and return to the data entry task on time. They postponed interruptions until a more convenient moment in the data entry task, rather than switching often and addressing an information need as it emerged. 

%The final part of the thesis then reports two studies that explore if design interventions can make people adopt more desirable strategies, i.e. strategies that minimise switching between entering data and looking up the required data. 

This thesis demonstrates how looking up information influences people's data entry work, by testing the effect of information access costs on people's switching strategies between looking up and entering data, and evaluating how making information access cost more salient can influence their behaviour. It makes a theoretical contribution by showing how people adapt to small changes in information access costs not only by changing the number and duration of switches, but also the scheduling of these switches during the main task. It makes a practical contribution by showing how making information access costs more salient influences people's switching behaviour, and can help people make their switches shorter, and schedule them at more convenient moments during a task.

%The primary contribution of this thesis is to show the fragmented nature of an expenses task, and that the IAC of required information sources affects how people manage how they look up certain information, which in turns impacts accuracy and efficiency. This finding has implications for the design of the current expenses system, which are validated through demonstration of how changing certain design features can potentially better support people in managing these subtasks of looking up information for a data entry task.