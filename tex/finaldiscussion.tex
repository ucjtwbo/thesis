\chapter{General Discussion}\label{ch:Discussion}

\begin{mynote}
\subsubsection{Chapter outline}
This chapter summarises the research findings of this thesis. It discusses the contribution that the findings make to knowledge, the practical implications, discusses limitations and suggestions for future work.
\end{mynote}

The aim of this thesis is to understand how people can better manage the time spent on inquiries needed for a data entry task, with the goal to reduce the disruptiveness of inquiries for the task. The results show that time costs can be used to encourage users to keep inquiries short, reduce the number of inquiries, and postpone them until a more convenient moment in the task. In this chapter, I first summarise the findings of each study. I then discuss the contribution of this thesis and the practical implications for tools aimed to help workers manage their interruptions in the workplace. Lastly, I discuss any limitations, outstanding questions, and how these could be addressed in future work. 

\section{Summary of findings}
Chapter \ref{ch:12} aimed to answer the first research question:

\begin{enumerate}
\item
How do people manage inquiries for data entry in an office setting?
\end{enumerate}

Using an interview study and contextual inquiry, I demonstrated that people manage physical inquiries by postponing them until a convenient moment in the task if they are expected to take time. Digital inquiries are managed by addressing them immediately as these are presumed to be quick to deal with.
%reported an interview study and a contextual inquiry study looking at the context in which office workers in finance offices conduct data entry work, and how they self-interrupt to look up task-related information during this work. 
The studies made three contributions. First, there had previously not been many studies that tried to understand data entry behaviour in the uncontrolled setting of an office workplace. Interview findings from Study \hyperref[st:Study1]{1} revealed that a critical component of this type of work is not just entering the data, but collecting this information from various sources distributed in the physical and digital work environment. Second, the study showed that inquiries were handled differently than task-irrelevant interruptions: people try to avoid task-irrelevant interruptions, but have to make inquiries as part of data entry tasks. Third, observations in Study \hyperref[st:Study2]{2} revealed that there was a difference in how physical and digital information is collected. Participants were aware that physical sources take time to access and participants therefore prepared it beforehand, or postponed retrieving it until a more convenient moment in the task. However, computer window switches during the task were commonly observed as workers often realised during the task that they needed additional information, and they presumed these switches to be quick. These switches often took longer than intended, and participants were observed being logged out of the entry system, resuming the wrong data entry task, and reported it took time to resume their work after these longer switches. The hypothesis was made that the expected time it takes to collect information played an important part in how people address self-interruptions, and that workers were unaware of the time they actually spend on digital interruptions. Though these qualitative studies gave a better understanding of how data entry work is situated in an office setting and people's self-interruption strategies during this work, the extent to which these strategies were influenced by time costs is unclear. Therefore, a series of lab experiments was carried out to study the effect of time costs on people's switching behaviour in a controlled setting.

Chapter \ref{ch:34} reported three controlled experiments aimed to answer the second research question:

\begin{enumerate}
\setcounter{enumi}{1}
\item
Do time costs affect the number, duration and timing of inquiries? 
\end{enumerate}

Study 3 showed that participants reduce the number of these inquiries, and increase the duration of them. Study 4 and 5 demonstrated how time costs affect the timing of inquiries: high time cost inquiries are postponed until later in the task, and inquiries with a low time cost are addressed first.
%to test the hypothesis that people prioritise collecting information according to time costs: they first switch to information with low time costs, and postpone collecting information with higher time costs. 
These studies showed that, in a controlled setting where participants can learn the time costs involved in accessing information, they first switch to information sources that are fast to access, and switch more frequently to these sources. On the other hand, people either prepare or postpone looking up information which takes time. Study \hyperref[st:Study3]{3} showed that if people retrieve all data from the same source, they will reduce switches between entering and looking up data if the access costs to this source increases. As it took more time to access, offloading behaviour was observed as well, and several participants prepared items they were going to need nearby, but did not use them yet. Study \hyperref[st:Study4]{4} further demonstrates that when people have to retrieve data from multiple sources, they collect and group items that are quick to access first, and leave items that take longer to access until the end. Study \hyperref[st:Study5]{5} demonstrated the robustness of the effect of time costs in a multi-task setup: when dealing with two data entry tasks, people still entered items with a low time cost first, and interleaved between tasks to enter items with low costs first. As a result, participants made more omission errors and submitted tasks before they had completed entering all the items. These studies contribute to our understanding of the effect of time costs on self-interruption behaviour to collect information: if people know the expected time duration of an interruption, they make fewer interruptions that are long and postpone these switches. However, what remained unclear after these studies was whether people can learn time costs in a naturalistic setting. An issue with inquiries is that time costs are not always predictable: there are various opportunities to get distracted, and people may spend a longer time than they think or expect looking for information.

Chapter \ref{ch:56} focused on answering the third research question:

\begin{enumerate}
\setcounter{enumi}{2}
\item
Does time feedback affect the number, duration and awareness of time spent on inquiries for data entry?
\end{enumerate}

Study 6 showed that time feedback reduces the duration of inquiries, and had no effect on the number of inquiries. Study 7 showed that time feedback increases awareness of time spent on inquiries, which was used to reflect on what they were doing during an interruption. 
This part of the thesis reported the development and evaluation of a browser notification that showed people the average time they spend away from data entry work. It included an online experiment and a field study looking at whether making people more aware of time costs can be effective in managing self-interruptions outside a laboratory setting. These studies contribute to our understanding of how time feedback can help people reflect on actions during, and reduce the length of, their interruptions. Study \hyperref[st:Study6]{6} found that using an experimental data entry task, people who were shown how long they were away for made shorter window switches, were faster to complete the task and made fewer data entry errors. Study \hyperref[st:Study7]{7} evaluated the intervention with office workers processing expenses. Data from post-study interviews indicated that time feedback made participants reflect on what they were doing during interruptions. They avoided interruptions that were not relevant, and tried to avoid distractions during interruptions that were relevant. 

\section{Contributions}
The findings contribute to our understanding of how time costs influence self-interruptions, and how information about time costs can help users self-regulate their interruptions. I first discuss the contributions this thesis makes to knowledge. I then discuss the practical implications these findings may have to inform the design of future studies as well as interruption management tools. 

%In order to develop effective interruption management tools, it is important to understand what factors impact people’s self-interruption behaviour, and what type of interventions are most effective in making people adopt more desirable interruption behaviour. My work contributes to interruptions and self-regulation literature by showing how time costs influence self-interruption behaviour, and how information about the time costs of interruptions can help users self-regulate their interruptions. 



%The main contribution of this thesis is an increased understanding of the effect of time costs on people’s self-interruption behaviour to collect task-related information. 

%Contribution to knowledge: expected time effort affects self-itnerruption behaviour: people address them immediately if short
%Contribution to practice: showing people how long they go away for helps reduce length and 

\subsection{Contribution to knowledge}
\textbf{C1 - Time costs affect self-interruption behaviour}

The first contribution is an increased understanding of the effect of time costs on people’s self-interruption behaviour to collect task-related information. Prior work has shown that when people have to access information and the time cost to access this information increases, they reduce the number of switches, and increase the length of each switch \citep{Gray2006}. Chapter \ref{ch:34} extends this work and shows that if people deal with different time costs, they postpone switching to interruptions with the highest time costs. 

\textbf{C2 – Inquiries are handled differently than task-irrelevant self-interruptions}

The second contribution is showing that that inquiries are handled differently than task-irrelevant interruptions. Prior work has shown that task-irrelevant interruptions are more disruptive than relevant ones \citep{Iqbal2008}. The studies in this thesis indicate individual differences between the ability to self-regulate task-irrelevant interruptions: whereas office workers in Study \hyperref[st:Study1]{1} and \hyperref[st:Study2]{2} were fairly good at avoiding task-irrelevant interruptions during data entry work, office workers in Study \hyperref[st:Study7]{7} did address some interruptions if they were presumed to be quick to complete. However, it is often the relevant interruptions that are needed for work that can be problematic for all workers: these cannot be avoided, are predominantly expected to be quick and easy, but can nevertheless be disruptive, as they can end up being time-consuming and there are various opportunities to get distracted. 
%What the studies in Chapter 3 showed was that during data entry work, which requires focused attention, office workers can be fairly good at avoiding these task-irrelevant interruptions. 

\textbf{C3 – Demonstration how time feedback helps reduce time spent on interruptions}

The third contribution is showing how feedback on time spent on interruptions helps people reflect on actions during, and reduce the length of, their interruptions. Prior work has shown that people like to be in control of their own interruptions, and do not simply want to have distractions blocked \citep{Mark2018}, but that they often do not know what action to take based on reflective data \citep{Collins2014, Whittaker2016}. Chapter \ref{ch:56} showed that people are able to take action when they are given feedback on the length of their interruptions. Increasingly more applications are showing people how they spend their time in their applications, to better manage time on work-unrelated purposes \citep{Constine2018, Constine2018a, Lynley2018}. This thesis shows that time feedback can not only help in managing work unrelated interruptions, but also work-necessary interruptions to look up information.

\textbf{C4 – Data retrieval as a part of data entry research}

A fourth contribution is that it highlights that for some types of data entry work, a major component of the task is collecting data from various locations in the physical and digital work environment. This has often been overlooked in previous data entry work, and may have an impact on data entry performance: it can slow people down and increase the likelihood of errors. If data entry interfaces are to be used in situations where information is not readily available, they should be evaluated  by requiring participants to collect data from the environment.

\subsection{Practical implications}
This thesis makes a practical contribution by demonstrating that giving people feedback on the length of their interruptions influences their interruption behaviour: through a better awareness of the length of their interruptions, users reflect on what they were doing during these interruptions, and where possible tried to make them shorter. In this thesis, I focused on a particular type of work and setting: data entry work in financial administration offices. Managing self-interruptions however is not just important for this particular setting, as interruptions and distractions are common in many kinds of computer-based work \citep{Gonzalez2004}. Below I discuss the practical implications of my findings which can inform the design of future studies as well as the design and evaluation of interruption management technologies. 

\textbf{P1 - Regulating distracting but work-relevant interruptions}. 

A common approach to improve task focus is to block any sources that are considered distracting from the primary task \citep{Kim2017, Mark2018}. However, this blocking approach centralises on the type of information source, but not the type of interruption: an information source may be distracting, but the interruption to access this source may be necessary for a task. Study \hyperref[st:Study2]{2} and \hyperref[st:Study7]{7} showed that people often need information sources they consider distracting, such as email, and cannot block these. This means we not only need to consider blocking interruptions that may be distracting from work, but also what support people can be given to control interruptions which are needed for, and considered part of, the task they want to focus on. Because it is difficult for a tool to determine why interruptions are being made (and whether they are necessary or not), it is better to give users tools to help self-regulate interruptions themselves. 

%Interruptions are made for various reasons, and while some may be necessary to progress with work, some are not and are best avoided. 

\textbf{P2 - Making time spent more visible}. 

The studies in this thesis have shown that people try to avoid long interruptions, which further extends prior research showing how people try to minimise time, and are sensitive to milliseconds \citep{Charman2003, Gray2004}. This thesis has shown however that outside of a controlled setting, these milliseconds may not be that visible. Given the thesis finding that people do adapt to time if they are made aware of time costs, there is therefore a need to make it explicit to people how long certain actions take for them to be able to adapt their behaviour.

\textbf{P3 - Differentiate between different types of self-interruptions}. 

This thesis focused on inquiries, a particular type of interruption, and Study \hyperref[st:Study2]{2} and \hyperref[st:Study7]{7} showed that people address this interruption differently than interruptions that are completely unrelated to their current task. Participants in these studies tried to avoid work-irrelevant interruptions, but work-related interruptions were addressed immediately, if they were presumed to be quick and easy. When discussing and making conclusions about people's self-interruption behaviour, it is important to make a distinction between different types of interruptions.

\textbf{P4 - Data retrieval as a part of data entry research}. 

Prior data entry research has primarily focused on improving entry interfaces. For the type of data entry work studied in this thesis, a major part is collecting data in the first place from various locations, and the entry part is actually only a small part of the task. This has implications for future data entry research, as it highlights that more attention needs to be given to the retrieval aspect of a data entry task, which impacts how data is entered. Study \hyperref[st:Study1]{1} showed that office workers tried to batch as much data entry tasks together as possible. A consequence of this was that they had to enter large amounts of data, and did not check their data entries as carefully as when they were checking their own work. Furthermore, when switching between documents, people held items in memory which increases the likelihood of error when returning to the data entry interface. Lastly, Study \hyperref[st:Study2]{2} and \hyperref[st:Study6]{6} highlighted that people often spent long times away from the data entry interface, before they returned. Data entry interfaces should take this into consideration and make it easier to resume a task. If data entry interfaces are intended to be used in situations where information is not readily available, they should be evaluated by requiring participants to first collect data from the environment, to see how usable they are in this context.

Prior to this thesis, there was no suitable data entry task to evaluate data collection from different sources from the task environment. As part of this thesis I developed a new experimental task, which can be used in future data entry studies to investigate time costs of collecting data during a data entry task.

\section{Future work}
My work contributes to our knowledge of the effect of time costs and feedback on self-interruptions, and introduces new opportunities that, building on its findings, further investigates how time costs can be utilised to effectively support self-interruption management.

%My work opens up areas for future research, and there are remaining questions of how the design intervention introduced in this thesis would be used over time, and in other settings. 

%First, the browser notification was simple in design. The aim of the last two studies was to test the effect of time feedback on duration of interruptions and task performance, rather than studying people’s engagement with it over time. Future work would be to further refine the implementation and 

\subsection{Complementing with other solutions}
One area to explore further is how the tool presented in this thesis can be combined and complemented with other approaches. For example, prior work has looked at different approaches to reduce disruptiveness by blocking interruptions and giving reflective information. In Chapter \ref{ch:56} the limitations of these interventions were discussed. The browser notification I developed as part of this thesis addressed some of these limitations. The browser notification did not block anything but specifically gave information on the duration of interruptions, as this was found to be an important deciding factor in people's self-interruption behaviour. It would be interesting to explore how different interventions could be combined and complement each other. For example, in Study \hyperref[st:Study7]{7} one effect of the browser notification was that it made people reflect on their actions during interruptions. The browser notification may work as an initial trigger, but could be complemented with a more extensive activity log such as those provided by RescueTime and ManicTime, so people are able to investigate what they were doing during an interruption, and why some interruptions may be longer than others. Prior work has shown that a barrier for people to currently engage with it is that it is not clear what to do with the extensive data and that it lacks context \citep{Collins2014}. Future work would need to investigate whether giving people a trigger may help as an entry point to better explore, understand and use  reflective data. 

\subsection{Time feedback to improve task performance}
The main research question of this thesis was: how can interruption management tools support people in managing inquiries for a routine data entry task, given variable time costs of required inquiries? In this thesis, I argued that increasing awareness of time spent on interruptions may support people in managing inquiries, and result in better performance. 
The potential for better performance was demonstrated in Study \hyperref[st:Study6]{6} and \hyperref[st:Study7]{7}: participants in Study \hyperref[st:Study6]{6} who were shown how long they were away for made shorter interruptions, were faster to complete the task and made fewer errors. While no quantitative performance metrics were collected in Study \hyperref[st:Study7]{7} to measure work productivity, participants in Study \hyperref[st:Study7]{7} explained that time feedback made them reflect on what they were doing during an interruption. As a result, they tried to be more focused on their work and were more wary of potential distractions.

Measuring productivity in the workplace has been known to be a difficult issue \citep{Mark2015}. To address this issue, I used an experimental task in Study \hyperref[st:Study6]{6} to objectively measure task performance, and interviews in Study \hyperref[st:Study7]{7} to gather data on people’s perceived productivity. Measuring productivity however remains an important issue that would be a valuable topic for future work. A next step in evaluating the intervention would be to further explore whether the impact of the intervention can be measured on real-world behaviour, and if people become more productive as a result of receiving time information. 

\subsection{Predicting the type of interruption to advise length of interruption}
The longer people interrupt, the more disruptive it can be, and a longer duration can indicate that people are getting distracted. However, as became apparent in Study \hyperref[st:Study7]{7}, sometimes people need to make a long interruption. In these cases, the duration may not be informative about whether people are getting distracted or not, and may not be useful as people are limited in their ability to shorten the duration of these interruptions. Future work could investigate people's actions during an interruption, to see whether people are making a necessary and useful interruption, or whether they are getting distracted. To this end, people's interactions during an interruption could be explored as a measure. For example, the browser extension DataSelfie\footnote{https://dataselfie.it/} tracks users' interactions when switching to Facebook to try and predict what the user is doing: are they actively looking something up to return to their work, or are they browsing through feeds? This information may get used to advise people on an appropriate time length, and give them timely reminders to return to a task. 

%Looking at the source people switch to is not sufficient, as people switch to distracting sources both for work and leisure.

%Not all self-interruptions are equally bad: prior work has investigated the timing, relevance, and workload of an interruption on task performance. This thesis has mainly focused on the duration of an interruption, as this was found to be an important deciding factor in people's decisions when and whether to self-interrupt. Furthermore, t

\subsection{Tracking behaviour over time}
%Lastly, further work is needed to determine how people engage with time feedback over time. 
This thesis ended with two formative studies exploring the use of a browser notification for about 15 minutes in Study \hyperref[st:Study6]{6}, and one working week during Study \hyperref[st:Study7]{7}. Though the results are promising, a remaining question is whether people would continue to engage with the notification over time, as many personal informatics tools get abandoned \citep{Lazar2015}. 

Furthermore, the notification only gave feedback on people’s current interruption behaviour. Some participants in Study \hyperref[st:Study7]{7} commented that they would like to see an overview of all interruptions, to see whether the actions they are taking has any considerable effect, and not just on their interruptions at that moment.

\subsection{Cross-device time feedback}
In its current implementation, the notification was evaluated using browser-based work, and intended for managing switching windows on this same device. People increasingly use multiple devices and may switch between devices to look up information on their phone or tablet \citep{Dearman2008, Jokela2015a, Murphy2016}. It would be easy to imagine to extend the notification across devices and take into account cross-device actions, which can show people a notification on a new device they switch to. Furthermore, it may be implemented as a standalone application to provide interruption information when doing a task in a non-browser computer window, such as when writing a document in a word processor.

\section{Generalisability of findings}

\subsection{Generalising findings to other types of interruptions}
In this thesis, I focused on a particular type of interruption, inquiries. In addition to inquiries, people can experience various other types of interruptions during computer-based work, such as getting interrupted by a phone call, or taking a break. Different types of interruptions have different triggers, and it was therefore important to pose restrictions on the scope of the thesis. In this section I discuss the generalisability of the thesis findings to other types of interruptions. 

Study 1 and 2 showed that inquiries are handled differently by people than task-irrelevant self-interruptions. Whereas participants overall tried to avoid task-irrelevant interruptions, inquiries have to be addressed in order to progress with work. There are however some external interruptions that may be unrelated to the task, but need to be addressed because of its urgency or importance, such as getting a request by a senior manager. In this situation, addressing the interruption may be more important than managing the time you are away from the current task. I therefore expect the findings of this thesis to generalise to self-interruptions triggered by the current task, that users are in control of managing themselves. Taking Jin \& Dabbish’s taxonomy, I expect the findings to generalize to the following types of self-interruptions: adjustments, where the user goes to adjust the task environment to optimize work; routines, such as checking email for work; and waits, when a delay in the current task motivates users to do something else. I do not expect the findings to generalise to triggers or recollections, because in these cases the user may switch to a different task in favour of the current task.

Even though the proposed intervention was designed with the aim to support inquiries, it can be used for all types of self-interruptions. 

\subsection{Extending research findings to other settings}
The type of task studied throughout this thesis revolved around a main data entry interface and was characterised by switching frequently and, usually, for short amounts of time. As such, information seeking was considered as a subtask to support the primary task and not analysed as a task in itself. 
I expect the results of this thesis to generalise to similar types of desktop-based work, where the user does not know beforehand which information is needed and has to make frequent and short switches to many different information sources. 

In other domains and types of work, the balance between information seeking and information use may be different \citep{Bondarenko2005}: for example, information workers in law offices have to spend a large proportion of their time seeking information \citep{Cangiano2009}, and may spend approximately equal amounts of time in several different computer windows. \citet{Bondarenko2005} define these as 'research tasks', and states that in contrast to administrative tasks, people make fewer interruptions during these tasks, often have to re-find previously used information, and spend a longer time in documents. The results of this thesis may not extend to these types of tasks, and in this context it may be of less importance how long and how often people look for information, but rather how they can be supported where to find information. A research area briefly discussed in Chapter \ref{ch:56} was to make it easier to collect and keep information nearby, reducing the duration of interruptions. As data entry work often requires new information, this idea was not developed further in this thesis, but it may be worthwhile to study for research tasks. It would also be interesting to explore how people address inquiries in these different domains. 

%In this thesis, I focused on a particular type of work in a specific setting: data entry work in financial administration offices. What was interesting about this type of work is that interruptions can be especially disruptive as they increase errors, but at the same time it is a type of task where interruptions are necessary. This task characteristic is not unique for data entry work in offices in particular, and there many types of computer-based tasks that involve frequent switching between computer windows \citep{Czerwinski2004}. 

\subsection{Limitations}
Limitations which are particular to a study have been discussed in previous chapters, but here I discuss a number of limitations that concern the methodology of the thesis overall. 

\subsubsection{Manipulating a qualitative phenomenon in a quantitative way}
This thesis used a mixed method approach to both understand an underlying mechanism of observed real-world behaviour in a controlled setting, as well as understand the generalisability of experimental findings in a naturalistic setting. Based on observational findings of Study \hyperref[st:Study2]{2}, I made the hypothesis that people either address or postpone inquiries because of time costs, which I tested through a series of controlled experiments. However, a limitation of this approach is that by converting a qualitative finding into a quantitative variable and controlling for other possible confounds, there is a disadvantage in it not accurately reflecting the actual phenomenon observed \citep{Driscoll2007}. While the manipulation of time costs in Chapter \ref{ch:34} is a simplified version of all the time costs involved to retrieve information in data entry work (e.g. physical effort to retrieve information from a location, time spent searching information, time spent opening and loading information sources), the findings in Chapter \ref{ch:34} are supported by existing literature and findings from Chapter \ref{ch:12}. Overall, the findings were consistent across qualitative and quantitative studies, and the manipulation was deemed appropriate as a first step to expand our understanding of how time costs contribute to self-interruption behaviour. Other time costs, such as physical effort \citep{Potts2017}, may affect task strategies differently but were beyond the scope of this thesis. 

%This gives a good reason to believe that qualitative behaviour is at least partly due to time costs, though I acknowledge this explains only part of the complex picture and there are many more underlying mechanisms to explain the behaviour. [reliable, effective?] 

%Furthermore, there were limitations in gathering sufficient quantitative data in the qualitative studies (Study 1, 2, 6), and a large part of the data in these studies is based on self-reports from participants, and observational data from the researcher. quantitative data in the experiments largely corroborates the qualitative findings,  but these different types of data are drawn from different participants doing similar, but not the same work.  The results from the different studies were consistent and quantitative and qualitative data complemented each other in understanding how time costs has an effect on task performance, self-interruption behaviour, as well as people’s subjective experience.

\subsubsection{Different participant populations}
In addition, participants of different studies were drawn from different populations: the participants in the qualitative Studies \hyperref[st:Study1]{1}, \hyperref[st:Study2]{2}, and \hyperref[st:Study7]{7} were office workers, while participants in the quantitative Studies 3-6 were from a range of backgrounds. The reason for not exclusively recruiting office workers for the experimental studies as well was that to make significant claims on the effect of time costs and time feedback on people's strategies, it was important to have a sufficient sample size. Office workers have busy work schedules and were hard to recruit, which is also reflected in the relatively small sample size for the qualitative studies. Participation for the experimental studies was therefore opened up to other participants as well. The benefit of combining workplace studies with experiments was that the research findings of this thesis can be generalised beyond a specific office setting: self-interruptions are common in many computer-based tasks, which makes the research findings not just relevant for office workers. Nevertheless, user expertise and job experience may influence people's strategies \citep{Weir2007}, and future studies could look more into the extent of which expertise contributes to the development of specific strategies. 

\subsubsection{Focus on digital inquiries}
This thesis focused on inquiries, a particular type of self-interruption triggered by the need to look up task information. Different types of interruptions have different triggers and may need different support. For instance, a routine to check social media may be non-essential to work and can thus be blocked temporarily, whereas inquiries are necessary to progress with the task. It was therefore important to pose restrictions on the scope of the thesis, to design a suitable intervention and make a valuable contribution. 

Though the intervention was designed with the aim to support managing inquiries, it showed time feedback for all task interruptions and can be used to manage other interruptions as well. For example, participants in Study \hyperref[st:Study7]{7} mentioned that the intervention also made them consider whether queries by colleagues were relevant to address at that moment.

The thesis also primarily focused on digital inquiries, and not physical inquiries. Study \hyperref[st:Study2]{2} highlighted that in particular assessing the time spent on digital inquiries was an issue for users, and that digital and physical inquiries were handled differently. Physical inquiries were largely planned for, but digital inquiries were often addressed immediately. Study \hyperref[st:Study7]{7} did suggest that even during physical inquiries, there was the potential to get distracted and that these could take longer than intended. Future work could evaluate whether showing time feedback on non-digital inquiries has a similar effect of reducing time spent on interruptions.

%Furthermore, the majority of data gathered at offices was qualitative. It would be useful to conduct future studies at office settings that will allow for additional quantitative data gathering techniques.

%[Todo: concluding sentence,  refer to literature on participant generalisability]

%Participation was limited to office workers in the following studies: it was essential to recruit office workers in Study 1 and 2, as the aim of these studies was to get a detailed understanding of how data entry work is situated in an office context. Furthermore, it was important to recruit office workers in Study 7, to get an understanding how appropriate the developed intervention would be in this context. 

%Practical implication. Prior work on self-interruption management has mostly considered self-interruptions which are not necessary, and thus can be controlled by users if they are given the right tools to self-regulate. The results in this thesis suggest that these existing approaches to interruption management are insufficient and inappropriate for necessary, task-related interruptions. 

\section{Conclusion}
Many computer-based tasks require users to interrupt themselves to collect information, which can lead to distractions, and it is challenging to remain focused on the primary task. Prior research has shown that the longer an interruption, the slower people are to resume and the higher likelihood of errors being made. The work in this thesis shows that presumed time effort affects how people address these types of interruptions: they address interruptions immediately if they except them to be short. The implication of this is that people may make many interruptions they think they can return from quickly - even if these end up being far longer than intended.

This thesis has also shown how making it explicit to people how long they actually go away from a task can help in managing self-interruptions: making people aware triggers them to reflect on what they are doing during these interruptions, and makes them reduce the length of interruptions. This is an important finding which has implications for the design of interruption management and productivity tools. %It opens up a lot of interesting questions which would be worthwhile to address in future work. 

My work makes an important contribution to interruptions literature. Prior work has shown that interruptions become more disruptive the longer they are. By demonstrating in this thesis that time costs influences people’s self-interruption behaviour, it highlights the need to make people more aware of when they are making interruptions and for how long. My findings extend our understanding of the factors impacting how people manage self-interruptions. 

%Even though I focused on a particular type of interruption triggered to look up information. 
