\chapter{Understanding the effect of time costs on inquiry strategies}

\begin{mynote}
\subsubsection{Chapter outline}
The previous chapter demonstrated that people adopt different strategies to address different inquiries. This chapter describes three controlled experiments that investigate the extent to which time costs of inquiries influence the number, duration and timing of inquiries for a data entry task. 

%Study 3 tests how IAC affects switching behaviour between one entry task and one information source. Study 4 tests switching behaviour between one entry task and multiple sources. Study 5 tests switching behaviour between multiple entry tasks and multiple sources.

Together these studies show that if the time cost of inquiries can be learnt in a controlled setting, participants try to minimise time by reducing the number of inquiries with a high time cost, and postponing these to be addressed later, rather than addressing them immediately. 

\end{mynote}
 
 \section{Chapter Introduction}
The previous chapter revealed that office workers have different interruption strategies to address inquiries, and that there are different time costs associated with these inquiries. If information had to be retrieved from another physical location, participants postponed to access it later. However, digital interruptions were addressed immediately, as they were presumed to be quick. These interruptions could often take longer than intended, which suggests that people are not aware of time spent on digital interruptions. In the Discussion section of Study 2, it was discussed that making people more aware of time costs may help them in better managing their interruptions. However, it is not clear from the qualitative studies alone whether time costs actually influenced people’s strategies, and what effect this has on task performance. This is important to understand, in order to know whether time information would be effective in managing inquiries and reducing its disruptiveness.

The aim of the studies reported in this chapter is to understand the effect of time costs on people’s inquiry strategies for a data entry task. In particular, the aim of Study 3 was to investigate the effect of time costs on the \textit{number} and \textit{duration} of inquiries and task performance. The aim of Study 4 was to investigate the effect of time costs on the \textit{timing} of inquiries. The aim of Study 5 was to investigate the effect of time costs on timing of inquiries in a \textit{multi-task setup}.

To address these aims, the studies use controlled laboratory experiments, which are a useful method to measure the effect of changes of one variable on another variable \citep{Cairns2008}. To be able to study the effect of time costs, it was necessary to simplify the complexity of various time costs observed in Study 2 into a single independent variable that could easily be manipulated in a controlled environment. In this chapter, time costs are manipulated as the time effort to access task information: in each experiment, participants were given task information and had to copy this information into a data entry interface. Time costs were manipulated by including a time delay to reveal and perceive the information to-be-copied. This manipulation enabled people to learn the time costs associated with inquiries through interaction with the interface. Furthermore, it is a type of time cost which has been used in prior research, making it straightforward to compare the study findings with prior studies.

\section{Study 3: Inquiries to a single source}\label{ch:Study3}

\textit{A subset of the data was collected by Katherine Corneilson, an affiliate undergraduate student at UCL, as part of an undergraduate project. I designed the study, collected the majority of the data and conducted the data analysis.}

\subsection{Introduction}
Participants in Study 1 and 2 had to retrieve and copy data from multiple sources with varying time costs for their work. Overall, participants were motivated to complete these tasks in an efficient manner: participants in Study 1 said they held information in memory when between different computer windows, rather than writing it down. Furthermore, if participants in Study 2 knew they were going to need certain information, and knew there was a high time cost associated with getting this information, they prepared the information before starting a data entry task. 

This is in line with the soft constraints hypothesis, a cognitive theory which states that people adapt their cognitive strategies to the constraints of a task environment with the aim to optimise task completion time \citep{Gray2006}. This theory has been tested through a series of lab experiments that manipulated the cost to access task information. These studies have consistently shown that as the cost to access task information increases, people try to minimise interruptions to access task information, and instead rely on information they have memorised. This may be a good strategy: making fewer interruptions is less disruptive, and people are quicker to resume the task after being away, because task information is still in memory \citep{Morgan2009}. However, the effect of this strategy on task performance differed across studies: while in some studies it made people more efficient and accurate, in others it slowed people down and increased errors. 

For example, \citet{Waldron2007} used a flight simulation task in their study, in which participants had to use flight information to navigate aircrafts to sites of interest. If there was a high time cost associated with accessing task information, participants increasingly relied on information in memory, and were more efficient to complete the task, as they did not have to interrupt their task as often to look back at the information. \citet{Gray2004} conducted an experiment where people had to copy over VCR programming information. Participants either had permanent access to the information, or they were explicitly instructed and trained before the trial to memorise the information. In the latter condition, the information during the trial was covered by a grey box which could be uncovered by hovering over it with the cursor. The people in the latter condition were more accurate than the other condition in entering the information. These studies showed that a memory-intensive strategy to save time can improve task performance, if the information is well-encoded in memory.

However, other studies instead found a decrease of task performance when the cost to access task information was increased \citep{Gray2006, Morgan2009}. In these studies, the Blocks World Task (BWT) was used as a task paradigm, which requires people to copy a 3x3 pattern of coloured blocks, by dragging blocks from a resource window to a target window and putting them in the correct order. They manipulated the cost to access task information across three conditions. In the Low Cost condition, the pattern was permanently visible on the screen. In the Medium Cost condition, the pattern was covered by a grey mask and participants had to hover over the mask with their mouse to reveal the pattern. In the High Cost condition, there was a 2.5 second delay before the pattern was revealed. As in prior studies, participants made fewer interruptions to access information if there was a time cost associated, and instead relied on information in memory. Looking at overall task performance however, they made more errors and took considerably longer to complete the task. In a later paper, the researchers reflected that the coloured blocks participants had to copy may have been too demanding to memorise \citep{Waldron2011}. This abstract visuo-spatial information did not bear any meaning to the participant, in contrast with the VCR programming information used in \citet{Gray2004} and the flight information used in \citet{Waldron2007} which resembles more familiar information used in a real-world task, and is easy to memorise.

This difference in results across studies suggests that the type of task information matters: if information is easy to remember, a memory-based strategy makes it better encoded in memory, with improved task performance as a result. However the studies not only differed in type of task information they used, but also in task paradigm, which makes it hard to compare their findings and say for certain the difference in results is due to the information: for instance, in \citet{Gray2004} people were explicitly instructed to memorise the information and completed a test prior to a trial during which they had to fill in the information, and could not continue until they had stated everything correctly. This ensured that people had the information well-memorised before they started the experimental trial. In the Blocks World Task studies, participants were not given this training or instruction.

It is important to not only understand to what extent people avoid time costs during a task, but also what effect this has on their task performance. Therefore Study 3 first replicates the BWT and only changes type of information, with the aim to see whether the effect of time costs on task performance depends on the type of task information. Study 3 is the only study in the thesis that uses this task paradigm, which was necessary to make a comparison with findings from previous work. The remainder of the thesis focuses on an expenses task which is based on the task observed in Study 1 and 2. 

\subsection{Method}
\subsubsection{Participants}
Fourty-two participants (eight male) were recruited from the UCL Psychology Subject Pool. Ages ranged from 18 to 52 with a mean age of 22.38 (SD = 7.45). Participants received course credit or \pounds3.75 as a compensation for taking part in the study.

\subsubsection{Materials}
Figure \ref{fig:ch4_taskparadigm} shows the task paradigm that was used. Each colour or number was only used once. The colours used were similar to the colours used in previous BWT studies \citep[e.g.][]{Gray2006, Morgan2009}.
Participants had to copy and complete fifteen patterns of each block type, and each participant had to copy over the same patterns. The target window showed a 3x3 grid with either coloured or numbered blocks. The output window showed an empty 3x3 grid, and was the same size as the target window. Participants had to copy the pattern shown in the target window by dragging blocks from the resource window and moving them into the output window. 

The study was conducted on a desktop computer, using a 24-inch monitor with a resolution of 2048 x 1152 pixels. Participants used a computer mouse to drag and drop blocks. The experimental task was implemented using HTML, Javascript and PHP and run in a browser.  All relevant browser events, such as mouse movements to (un)cover the grey mask, dragging and dropping the blocks and mouse clicks, were recorded and saved in a mySQL database. The browser window covered the whole screen to minimise distractions.

For the Low Cost condition, eye fixations were used to measure the number and duration of visits to the target window. Eyetracking data was also obtained for the Medium and High Cost conditions. However, this data was not used due to the fact that people were able to also view the target window area whilst the target pattern was covered. Therefore, in accordance with previous studies \citep[e.g.][]{Gray2004}, for the Medium and High Cost conditions the number and duration of uncovering the mask was taken as a measurement for visits to the target window.  These uncoverings were measured by Javascript. The usefulness and limitations of using these measures are discussed in the Discussion.

A Tobii T60 eyetracker was used for recording people's eye fixations. Eye movements were recorded at a rate of 60 gaze data points per second for each eye, with an accuracy of 0.5 degrees and timestamp accuracy of 4 ms. For the analysis, all consecutive eye fixations with no drag or drop actions in-between were added together and counted as one fixation.

\begin{figure}[]
\begin{center}

\begin{subfigure}[b]{\textwidth}
\centerline{\includegraphics[scale=0.23]{images/ch34/ch4_numbers.png}}
\caption{The number condition.}
\label{fig:ch4_BWT}
\end{subfigure}
%\hfill%
\begin{subfigure}[b]{0.5\textwidth}
\centerline{\includegraphics[scale=0.23]{images/ch34/ch4_colours.png}}
\caption{The colour condition.}
\label{fig:ch4_NWT}
\end{subfigure}
\caption[Study 3 task lay-out]{The task lay-out with the three different components.}
\label{fig:ch4_taskparadigm}
\end{center}
\end{figure}

\subsubsection{Design}
A mixed design was used with two independent variables: time cost and block type. The between-participants variable was the level of time cost which had three levels. If the Cost was Low, the target pattern was permanently visible. In the Medium and High Cost conditions, the target pattern was covered with a grey mask, and could only be uncovered by moving the mouse cursor over the window. The mask reappeared as soon as the cursor left the window. In the High Cost condition, there was an additional 1-second delay to uncover the mask. This delay time was used in previous BWT studies where it showed to have a significant effect on task strategies and performance \citep{Gray2006, Morgan2009, Waldron2007}. The within-participants variable was the block type to be copied, which was either coloured or numbered blocks. The order was counter-balanced across participants.

The dependent variables are listed in Table \ref{table:ch4_dvs}. The primary focus is on the measures of the first visit, as participants do not have any information yet on the target pattern. On subsequent visits, they may already have partial information in their head from previous visits. Therefore, the items copied after the first visit is believed to be the most 'sensitive measure of performance' \citep{Janssen2012}. 
Two dependent variables were used to measure accuracy. Incorrectly placed blocks measured instances where a participant initially placed a block in the incorrect place, but then moved this to the correct place prior to submitting the pattern. Incorrectly submitted trials measured instances where the participant had finished copying a pattern and clicked the Submit button, but the pattern was incorrect.

\begin{table}[htp]
\centering
    \begin{tabular}{  l }
    \hline
    \textbf{ Strategy measures} \\  
    Number of visits to target window  \\ 
    Visit duration of first visit (s)  \\
    Average duration of visits (s) \\
    Number of blocks copied after first visit \\
    Number of blocks copied correctly after first visit 
    
    \vspace{10pt} \\
    
\textbf{Global task performance measures} \\ 
Number of incorrectly placed blocks (per trial) \\
Number of incorrectly submitted trials (per experiment block) \\
Trial completion time incl. and excl. lockout (s) \\ \hline
    \end{tabular}
    \caption[Study 3 dependent variables]{Dependent variables used in the study.}
    \label{table:ch4_dvs}
\end{table}

\subsubsection{Procedure}
Participants were welcomed and briefed about the experiment. It was explained they would be shown nine blocks which were in a certain order, and had to copy this order by moving blocks around. Participants were instructed to complete the task as fast as possible, but it was explained that they were not able to continue until they had copied a pattern correctly. 

After the briefing, participants were asked to read and sign a consent form and were given an information sheet with a summary of the study and the researcher's contact details. In addition to the verbal briefing, the explanation of the study was written out on the computer screen for the participant to read and they were shown an instruction video that showed how the experiment worked. 
The experiment was broken down in two parts, one where they had to copy colours, and one where they had to copy numbers. For each part, they were given two practice trials first to get familiar with the set-up, and to give them a chance to ask questions if anything was unclear. There was an opportunity for the participant to take a break between the two parts. The study took around 20-30 minutes to complete.

\subsubsection{Ethical considerations}\label{sec:quanethics}
The study was undertaken with ethical approval from the UCL Research Ethics Committee [Project ID Number UCLIC/1415/001/Staff Brumby/Borghouts]. 
At the start of each study, participants were first briefed verbally about the study. They were asked to read and sign a consent form, and were given an information sheet to keep. This information sheet contained a summary of the study information and the researchers' contact details. It was explained that an eyetracker would record their eye fixations and movements, but that these recordings were anonymous and that they would not be directly identifiable. After participants had completed the first part of the experiment, a prompt appeared on the screen advising them to take a short break. Participants could take a break as long as they wanted and could decide themselves when to continue with the second part of the experiment.

Participants were informed that the data would be used for research purposes only and stored in accordance with the Data Protection Act 1998. They were also informed that their data would be anonymised and when used in a report or academic paper, their data would not be directly identifiable.

\subsection{Results}
The means and standard deviations of all dependent variables are shown in Table \ref{table:ch4_IACmeans}. Two-way mixed ANOVAs were used to analyse the effect of time cost and block type on the dependent variables. 

\subsubsection{Cleaning the data}
Eight participants were removed from the analysis due to weak eye-tracking calibration. Furthermore, one participant misunderstood the experiment and did not know she was allowed to uncover the mask of the target window more than once. This participant had scores that were more than three times the interquartile range from the rest of the participants' scores on six different variables, so this participant was considered an outlier and removed from the analysis. 

%\begin{table}
%\centering
%\ra{1.3}
%\begin{tabular}{|p{6cm}|lll|lll|}\toprule
\begin{tabular}{@{}p{6cm}lllclll@{}}\toprule %\begin{tabular}{  p{6cm} l p{1cm} l p{2cm} l  p{1cm}l p{1cm} | p{2cm} | p{1cm}|}
 & \multicolumn{3}{c}{\textbf{Colours}} & \multicolumn{3}{r}{\textbf{Numbers}} \\
 \cmidrule{2-4} \cmidrule{6-8}
 & Low & Medium & High && Low & Medium & High\\\midrule
\textbf{Strategy measures}\\
Number of visits to target window & \textbf{6.36} & \textbf{4.24} & \textbf{2.98} && \textbf{5.10} & \textbf{2.03} & \textbf{2.05} \\
						    & 		    2.28 & 		1.62   & 	          0.90 && 		2.48   & 	         0.63  & 	       0.67 \\
Visit time of first visit (s)  		    & \textbf{0.39} & \textbf{0.04} & \textbf{2.18} && \textbf{0.51} & \textbf{0.04} & \textbf{1.49} \\
				       		    &              0.23   & 		  0.02  & 	         1.59  && 		 0.45  & 	         0.05 & 	      1.01 \\
Average time of visits (s)  		    & \textbf{0.29} & \textbf{0.04} & \textbf{1.54} && \textbf{0.35} & \textbf{0.04} & \textbf{1.07} \\
				       		    &              0.13   & 		  0.02  & 	         0.95  && 		 0.15  & 	         0.03 & 	      0.77 \\
Number of blocks copied		    & \textbf{1.90} & \textbf{3.55} & \textbf{4.52} && \textbf{2.44} & \textbf{6.18} & \textbf{6.33} \\
 after first visit 		       		    &              1.84   & 		  1.93  & 	         1.43  && 		 1.89  & 	         1.61 & 	      1.67 \\
Number of blocks copied correctly & \textbf{1.86} & \textbf{3.22} & \textbf{4.07} && \textbf{2.36} & \textbf{5.96} & \textbf{5.98} \\
after first visit 		       		    &              1.75   & 		  1.83  & 	         1.20  && 		 1.74  & 	         1.52 & 	      1.55 \\

\vspace{10pt}

\textbf{Global task performance measures}\\
Number of incorrectly placed blocks 
				  		    & \textbf{0.15} & \textbf{0.67} & \textbf{0.79} && \textbf{0.17} & \textbf{0.31} & \textbf{0.46} \\
(per trial) 				       	    &              0.18   & 		  0.40  & 	         0.44  && 		 0.19  & 	         0.18 & 	      0.16 \\
Number of incorrectly submitted
				  		    & \textbf{0.27} & \textbf{1.9} & \textbf{2} && \textbf{0.36} & \textbf{0.5} & \textbf{0.83} \\
trials (per experiment block)	    &              0.65   & 		2.51 & 	   2.13  && 		 1.21  & 	         1.08 & 	      1.03 \\
Trial completion time incl. lockout
				  		    & \textbf{19.60} & \textbf{25.40} & \textbf{31.80} && \textbf{19.47} & \textbf{20.83} & \textbf{25.95} \\
(s)						    &                2.98   & 		5.16 & 	   	     6.08  && 	         3.03  & 	        3.08   & 	         4.21 \\
Trial completion time excl. lockout
				  		    & \textbf{19.60} & \textbf{25.40} & \textbf{28.84} && \textbf{19.47} & \textbf{20.83} & \textbf{23.89} \\
(s)						    &              2.98   & 		5.16 & 	   6.34  && 		 3.03  & 	         3.08 & 	      4.06 \\

\bottomrule
%\caption[Study 3 descriptive measures]{The effect of IAC on copying colours and numbers. The means are shown in bold, the standard deviations are below the means.}
\end{tabular}
\label{table:ch4_IACmeans}
%\end{table}

\subsubsection{Task strategies}

\subsubsection{Number of visits to the target window}
Participants made fewer visits to the target source when they had to copy numbers (M = 3.06, SD = 2.08) than when they had to copy colours (M = 4.49, SD = 2.18), F(1,30) = 41.62, p<.001, $\eta^2$  = 0.58. Participants also made fewer visits as IAC increased from Low (M = 5.73, SD = 2.41), to Med (M = 3.13, SD = 1.65), to High (M = 2.51, SD = 0.91), F(2,30) = 15.16, p<0.001, $\eta^2$  = 0.50. To investigate differences between conditions, post-hoc Tukey comparisons were performed. Results showed that participants made significantly fewer visits in the Medium-IAC condition than in the Low-IAC condition, p <.01. However, there was no difference in number of visits between the Medium-IAC and the High-IAC conditions, p=.59. Participants looked at the target window for colours more on every level of IAC (see Figure \ref{fig:ch4_noVisits}), and so there was no significant interaction, F(2,30) = 2.82, p=.08, $\eta^2$  = 0.16. 

\begin{figure}[!ht]
\centering
\includegraphics[width=\textwidth]{images/ch34/ch4_noVisits-bargraph.pdf}
\caption[Study 3 number of visits]{The interaction between block type and IAC for number of visits to the target window. The error bars represent $\pm $1 standard error.}
\vspace{-9pt}
\label{fig:ch4_noVisits}
\end{figure}

\subsubsection{Duration of first visit to target window}
There was no significant main effect of block type on the duration of the first visit, F(1,30) = 3.05, p=.09, $\eta^2$  = 0.09. Participants looked longer at the target source as IAC increased from Low to High. Post-hoc comparisons showed that participants looked longer in the High-IAC condition (M=1.84, SD = 1.35) than in the Low/Medium-IAC conditions, ps <.001. However, there was no difference in duration between the Low-IAC (M = 0.45, SD = 0.46) and the Medium-IAC (M = 0.05, SD = 0.04) conditions, p=.47. There was a significant interaction effect between IAC and block type, F(2,30) = 5.70, p<.01, $\eta^2$  = 0.28 (see Figure \ref{fig:ch4_firstVisitDuration}). There were no difference between block types in the Low-IAC condition, t(10) = -1.86, p = 0.09, nor the Medium-IAC condition, t(9) = -0.29, p = 0.7. However, in the High-IAC condition, participants looked significantly longer for colours (M = 2.18, SD = 1.59) than numbers (M = 1.49, SD = 1.01), t(11) = 2.76, p = 0.02.

\begin{figure}[!ht]
\centering
\includegraphics[width=\textwidth]{images/ch34/ch4_firstVisitDuration-bargraph.pdf}
\caption[Study 3 duration of first visit]{The effect of IAC on the duration of the first visit to the target window. The error bars represent $\pm $1 standard error.}
\vspace{-9pt}
\label{fig:ch4_firstVisitDuration}
\end{figure}

\subsubsection{Blocks placed after first visit}
People placed more blocks correctly after the first visit for numbers (M = 4.77, SD = 2.33) than colours (M = 3.08, SD = 1.81), F(1,30) = 63.86, p<.001, $\eta^2$  = 0.68. They also placed more blocks as IAC increased, F(2,30) = 12.54, p<0.001, $\eta^2$  = 0.46. Tukey post-hoc comparisons show there was a difference between the Low IAC and Medium/High IAC conditions (ps<.01), but not between Medium and High IAC conditions (p=.77). There was a significant interaction effect between IAC and block type, F(2,30) = 8.96, p<.01, $\eta^2$  = 0.37  (see Figure \ref{fig:ch4_firstCorrectBlocks}). When IAC was Low, the number of blocks that were copied correctly after the first visit did not differ significantly for colours or numbers.

\begin{figure}[!ht]
\centering
\includegraphics[width=\textwidth]{images/ch34/ch4_firstCorrectBlocks-bargraph.pdf}
\caption[Study 3 number of blocks correctly placed]{The interaction between block type and IAC for number of blocks correctly placed after the first visit to the target window. The error bars represent $\pm $1 standard error.}
\vspace{-9pt}
\label{fig:ch4_firstCorrectBlocks}
\end{figure}

\subsubsection{Global task performance}
The interactions between block type and IAC on global task performance measures all had the same trend: people performed the same for colours and numbers when IAC was Low, but differences appeared between the block types as IAC increased. As this trend was the same for each interaction, the statistical results of the interactions are reported but their specific trend will not be repeated.

\subsubsection{Number of incorrectly placed blocks}
Participants placed more blocks incorrectly for colours (M = 0.54, SD = 0.45) than numbers (M = 0.32, SD = 0.21), F(1,30) = 10.71, p=.003, $\eta^2$ = 0.26. As IAC increased and participants were keeping more items in memory, they increasingly placed more incorrect blocks, F(2,30) = 14.71, p<.001, $\eta^2$ = 0.50. Tukey post-hoc comparisons show there was a difference between the Low IAC condition (M = 0.16, SD = 0.18) and Medium/High IAC conditions (ps<.01), but not between the Medium (M = 0.49, SD = 0.35) and High IAC conditions (M = 0.63, SD = 0.36) (p = .3). There was a significant interaction effect between IAC and block type, F(2,30) = 3.36, p<.05, $\eta^2$ = 0.18. When IAC was Low, the number of blocks that were copied incorrectly did not differ significantly for colours or numbers, but as IAC increased, participants placed more blocks incorrectly for colours.

\subsubsection{Number of incorrectly submitted trials}
The number of trials that were submitted incorrectly was generally low, but participants submitted more incorrect trials for colours (M = 0.1, SD = 0.16) than numbers (M = 0.04, SD = 0.08), F(1,30) = 5.28, p=.029, $\eta^2$ = 0.15. There was no significant effect of IAC, F(2,30) = 2.70, p=0.08, $\eta^2$ = 0.15, nor any interaction, F(2,30) = 1.65, p=.2, $\eta^2$ = 0.10.

\subsubsection{Trial time}
Two trial completion times are considered here: total time and time excluding lockout. 
Looking at the actual completion time, participants took longer to complete a trial when they were copying colours (M = 25.80, SD = 7.06) compared to when copying numbers (M = 22.24, SD = 4.47), F(1,30) = 44.08, p<.001, $\eta^2$ = 0.60. As IAC increased from Low to Medium to High, participants took longer to complete a trial, IAC, F(2,30) = 15.91, p<0.001, $\eta^2$ = 0.52. Tukey post-hoc comparisons show there was a difference between Low/Medium and High (ps<.01), but not between Low and Medium (p = .12). There was a significant interaction effect between IAC and block type, F(2,30) = 11.05, p<.001, $\eta^2$ = 0.42. 

With the lockout time in the High-IAC condition removed, the same effects were found for block type, F(1,30) = 34.55, p<.001, $\eta^2$ = 0.54, and IAC, F(2,30) = 8.18, p=0.001, $\eta^2$ = 0.35. Tukey post-hoc comparisons show there was still a difference between Low and High (p=.001), but no longer between the Medium IAC and Low IAC or High IAC conditions (ps >.1). There was a significant interaction effect between IAC and block type, F(2,30) = 8.13, p=.002, $\eta^2$ = 0.35.

\subsubsection{Qualitative data}
The screen recordings from the eye-tracker were played back to further investigate people's behaviour. Although this helped understand some behaviour which could not be determined from the quantitative data alone, these observations only serve to explain some of the quantitative measures and are not the main focus of the analysis.

The visit durations in the Medium IAC condition were suspiciously short. Upon replaying the screen recordings, it appeared that participants often accidentally moved their cursor over the grey mask of the target source. This was counted as a visit by the program, even though participants may have not intentionally moved their cursor to this part of the screen to look at the target source. They did not spend a long time looking at the target window, but also did not immediately move blocks either, and sometimes waited multiple seconds before they made a move. 

\begin{figure}[!ht]
\centering
\includegraphics[scale=0.3]{images/ch34/ch4_placeholders.pdf}
\caption[Study 3 placeholders]{Participants placed blocks outside of the output window as `placeholders'.}
\vspace{-9pt}
\label{fig:ch4_placeholders}
\end{figure}

During the 1-s lockout in the High IAC condition, participants changed their minds about visiting the target window on numerous occasions. They placed their mouse cursor on the mask, but left this field before it was uncovered to move one or more blocks. It could be this decision also occurred in the Medium IAC condition, but as there was no lockout the mask was already uncovered before people made this decision, and would explain the very short visits.

People sometimes placed the blocks as `placeholders' as shown in Figure \ref{fig:ch4_placeholders}: they placed several blocks outside of the output window next to the position they thought it belonged to, but did not place it there yet. Only after viewing the target again, they placed the blocks in the output window. Looking at quantitative data alone, this type of strategy would be depicted as one long view at the target, after which all blocks were placed in one go. This is true to some extent, but as people could already place the blocks and offload their memory without this being recorded by the program, they only had to check if this position was correct on the subsequent visit, and is different from a strategy where people spent a long time trying to memorise the blocks after which all blocks were placed.

\newpage

\subsection{Discussion}
The aim of this study was to investigate the effect of time costs on the number and duration of inquiries, and the effect on task performance. The main findings are:

\begin{itemize}
\item
Increases in time costs make people adopt memory-intensive strategies
\item
The effect of a memory-intensive strategy on task performance depends on the type of information
\item
The effect of a memory-intensive strategy on task performance depends on the type of task
\end{itemize}

%The study replicated the BWT study to investigate whether the effect of time costs, as found in previous experimental studies, would extend to different types of information. We learn that people try to avoid time costs, and as time costs increase minimise switches to task information, and instead rely on information in their head. The study shows that the type of information matters: contrary to copying colours, when copying numbers this strategy did not increase any errors so may be a safe strategy, although it did not increase people’s performance either. 

%We also learn that the type of task matters: slowing people down, but not realistic. 

%Memory-intensive strategy
The effect of time costs on people's strategies is consistent with prior work \citep{Gray2006, Morgan2009, Waldron2007}. People switched from a perceptual to a memory-based strategy by making fewer but longer visits to the target window and placing more blocks immediately after the first visit. This further confirms that in a controlled setting, people are sensitive to small increases in time costs and try to avoid time costs.

In the colours condition, a memory-intensive strategy worsened participants' task performance, as they took longer to complete the task and placed more incorrect blocks throughout the trials. In the numbers condition, this strategy did not increase errors, which shows that the type of information matters in a task. Numbers were likely easier to memorise, which was demonstrated by the higher number of blocks that were copied after a first visit to the target window: on average, people placed six numbered blocks after a first visit, which is about the number of items people can hold in short-term memory \citep{Miller1956}. In comparison, people only placed on average three coloured blocks after a first visit. Numbers can be rehearsed, and therefore refreshed in working memory, whereas visuo-spatial information such as coloured blocks is more difficult to memorise \citep{Baddeley1974}. 

Contrary to prior work however \citep{Gray2004, Soboczenski2013}, a memory-based strategy when copying numbers did not make people more efficient or accurate, which could have been due to the nature of the task. The error rate was overall low and upon reflection the interaction of moving blocks may have made people sufficiently slow to hardly make any errors. In previous studies, people typed in data using a computer keyboard, in which it is more likely to make data entry errors due to slips \citep{Oladimeji2011}.
 
 \subsubsection{Limitations}
 The study used a similar manipulation of time costs as in previous BWT studies. Using this manipulation, it was difficult to measure visits to the target window in the same manner for all conditions. For the Low Cost conditions, eye fixations were used, whereas for the Medium and High Cost conditions, uncoverings of the mask were used. This introduced several problems. First, while eye-tracking measures show how long and how often people are looking at a particular part of the screen, it can not reveal if people are actually perceiving or processing the data that is displayed \citep{Waldron2007}. For the Medium and High Cost conditions, an interaction was required and a conscious decision had to be made to reveal the target in these conditions. It would therefore seem likely that uncoverings more reliably measure visits to the target window. However, the uncoverings for the Medium Cost conditions were suspiciously short. Playing back the screen recordings suggests participants often accidentally uncovered the window, and it is therefore less clear if these are a reliable measure of actual visits. In future studies, the setup of the experiment should be designed so that participants make a conscious decision to reveal the target information, but do not accidentally access the source when they do not intend to. In the next two experiments, the same consistent measure is used (i.e. a mouse click to make a window switch) to study inquiries across conditions. 
 
 \subsection{Conclusion}
The aim of this study was to study the effect of time costs on number and duration of inquiries and task performance. The results show that if people retrieve all data from the same source, they will reduce switches between entering and looking up data if the access costs to this source increases. As it took more time to access, offloading behaviour was observed as well, and several participants prepared items they were going to need nearby, but did not use them yet. 

The aim of this chapter is to investigate the effect of time costs on strategies. Study 3 learnt us that time costs reduce number of switches and increases duration of switches. The task however only involved one source, in contrast with the task studied in Study 1 and 2, where people had to deal with but various sources, all with different time costs. While we now have a better understanding of the effect of time costs on number and duration of inquiries, we do not know the effect on the timing of inquiries yet. This will be investigated in the next two studies.

%As a result, people not only have to make decisions on how often they make inquiries to these sources, but also when they interrupt their task to make inquiries.

\section{Study 4: Copying data from multiple sources}
 
\subsection{Introduction}
%Motivation

Study 3 showed people avoid time costs by making fewer inquiries to an information source. Participants tried to group and memorise as much information, in order to minimise the number of revisits to this source. In the experiment, all information was to be found on a single source. As discussed in Chapter 3 however, data entry in office workplaces often does not switching between a task and a single source, but information can be spread over various sources with different time costs associated with them. Information can be one click away, or time has to be spent accessing it. What we do not know from Study 3 is how time costs affect how people schedule inquiries to different sources with different time costs. Observational findings from Study 2 suggest that inquiries with a high time cost are postponed: participants prepared physical information sources either before starting work or postponed to access it later. However window switches during a task were commonly observed, as these were presumed to be quick. Furthermore, in prior work information access cost was found to be a main factor that determined whether participants looked up information on their mobile phone as soon as they needed it, or whether they postponed to address it later \citep{Sohn2008}. While these findings demonstrate people take time costs into account when accessing information on physical sources and mobile phones, it is unclear whether participants take time costs of switching windows on a desktop computer into account. Easy switching between windows on conventional desktop computers may give the false impression that information is easy to access \citep{Sellen2003}. 

The aim of Study 4 is to understand the effect of time costs on the timing of inquiries for a data entry task. An experiment was conducted in which participants had to complete a data entry task, and look up the to-be-entered items by switching to two different computer windows. While prior work has demonstrated that various tasks can involve the use of multiple information sources \citep{Cangiano2009, Murphy2016, Su2013}, it has not been measured how people access these sources, and to what extent the time cost to access a source influences these decisions. Based on the postpone strategies observed in Study 2, the following hypothesis is made:

\begin{itemize}
\item [H1.]
As the experiment progresses and people become aware how costly it is to access certain sources, they will learn to postpone entering High-Cost items, and choose to enter the Low-Cost items first. 
\end{itemize}

Prior work has shown that increased time costs encourage people to learn more efficient strategies, which they then transfer to use in other situations in which time costs are no longer high \citep{OHara1998, Patrick2014, Waldron2007}. For instance, \citet{Patrick2014} conducted an experiment where participants had to complete a Blocks World Task. Some participants had permanent access to the target pattern, whereas other participants had to complete a number of trials first, in which the target pattern was hard to access. People who were exposed to the interface with an increased access cost first adopted a memory-based strategy and retained this strategy, even when they then interacted with an interface with lower access costs. It is therefore expected that once participants learn it is more efficient to group High-Cost items, they may adopt this strategy for Low-Cost items as well:

\begin{itemize}
\item [H2.]

As the experiment progresses and people become aware how costly it is to access certain sources, participants in the High-Cost conditions will learn and choose to enter all the Low-Cost items in a batch, and then the High-Cost items in a batch, rather than looking up each item as they need it. 

\end{itemize}

\subsection{Method}
\subsubsection{Participants}
Thirty-three participants (12 male) ranging from 18-52 years (M = 26, SD= 8) took part in the experiment. They were recruited from a university subject pool and received $\pounds$4 for their participation.

\subsubsection{Task}
The experimental task was based on an expenses task, a routine data entry task observed in the studies in Chapter 3. For this task, the user has to complete a number of data entries regarding incurred expenses in order to get the expenses reimbursed. They enter this into a claim form, which looks similar to a spreadsheet. 

For each trial, participants were presented with a data entry sheet consisting of two expense claims (see Figure \ref{fig:ch34_4-tasklayout}). They had to complete each row by entering a financial amount to specify an expense that was made, and an account code to specify which account to use to reimburse the expense. They retrieved these data items by switching to two other windows. One window contained the amounts, and another window contained the account codes. The participant could go to a window by clicking on the corresponding name in the horizontal menu at the top of the screen. Only one window could be viewed at a time and covered the full screen. 

\begin{figure}
 \includegraphics[width=\textwidth]{images/ch34/ch34-4_Tasksequence.pdf}
    \caption{The data entry task. At the start of each trial, participants were presented with a data entry form with two expense claims, and had to enter four data items in a data entry form. The data items were retrieved from a separate Amounts page (Step 1) and an Accounts page (Step 2) and entered into the data entry form (Step 3).}\label{fig:ch34_4-tasklayout}
\end{figure}

\subsubsection{Materials}
The numbers to be entered were made to resemble values that are ecologically relevant to an expenses task. The account codes were similar to codes that are currently used by the universities studied in Chapter 3, and have a fixed length of six digits (e.g. 654273). The string of digits was random with no particular pattern. Amounts consisted of two digits on the integer part and two digits on the fraction part (e.g. 11.95). 

The experiment was conducted in a maximised web browser on a desktop computer with a 24-inch monitor and a resolution of 2048x1152 pixels. Participants used a computer mouse and number keypad, and it was not possible to copy and paste information. If the participant switched from the data entry form to another page and back, the cursor stayed in the same data entry field. The task interface was developed in HTML, CSS, JavaScript and PHP. All mouse clicks, key presses and timestamps were recorded using JavaScript.

\subsubsection{Design}
A between-participants design was used with one independent variable, the presence or absence of a time cost when switching to one of the information windows. In the Control condition, there were no costs in switching between any of the windows. In the High-Amount condition, there was a 2-s delay when opening the Amount window, and in the High-Account condition there was a 2-s delay when opening the Account window. There were no delays in switching back to the data entry form in any of the conditions. 

To investigate the timing of inquiries, the order in which participants entered the data items was analysed. On a trial-by-trial basis, the main dependent variable was whether people interleaved between expenses or not: did participants enter the data items in sequential order (i.e. enter one expense first, and then the second expense), or did they interleave between the two expenses to enter items from the same source first (i.e. enter all amounts first, and then all account codes)? Two values had to be entered for each expense: an amount and an account code. If participants entered the amount and account code of one expense before entering the other expense, this was considered a sequential order. If participants entered amounts of each expense first, followed by entering the account codes or vice versa, this was considered interleaving. All key presses were recorded to determine in which order data was entered. Window switches were recorded to capture the number and duration of switches to information windows. Other dependent variables were trial completion time and data entry error rate. In addition, the type of errors was analysed. 

\subsubsection{Procedure}
The experiment took place in a closed quiet room. It was explained to participants that the task involved entering expenses, and that for each trial they had to enter two expenses. They were not instructed to use a particular strategy, but it was explained it was important to complete all data entry fields before proceeding to the next trial, as they could not return as soon as they had pressed 'Submit'. There were no restrictions in the number or duration of times they could switch between pages, or the order in which they completed the trial. One trial consisted of two expenses, i.e. four data entries. Participants first completed two practice trials to familiarise themselves with the task, and were free to ask any questions; data from these trials were not included in the analysis. After that, the experimental session consisted of 50 trials, divided into 5 blocks of 10 trials. After each block, there was an opportunity for the participant to take a short break. A prompt appeared on the computer screen, and the recording time was paused. Participants could carry on with the experiment by pressing a button on the screen. For each block, a set of 20 different amounts and 20 different account codes were used. These sets were re-used for every block, so in total, each number was presented five times throughout a session. The experiment took approximately 30 minutes.

\subsubsection{Pilot study}
Two pilot studies were conducted with colleagues to test the experimental design. In particular, the pilot studies aimed to see if the length of the experiment was long enough for participants to learn and develop strategies, but not too long to tire the participant.

During the pilot studies, there was a scheduled break after every 5 trials. Both participants mentioned the break prompts happened too frequently, and experienced them as disruptive. They did not find the experiment too long. One participant could not remember which computer tabs had an increased time cost. As a result, he did not adapt his strategies according to anticipated time costs and kept entering the data items row by row. The second participant mentioned that the increased time costs definitely made her more careful in checking the numbers were correct. The participants were aware some of the numbers occurred more than once, but the numbers did not occur often enough to be able to memorise them. 

For the real experiments, the breaks were reduced to happen after every 10 trials. In addition, the names of information windows with an increased time cost were underlined in the horizontal menu. This visual feature was added to help users see more easily which windows had a delay.

\subsubsection{Data analysis of task strategies}
A bottom-up approach was taken to group and analyse people's strategies. For the first iteration of grouping, each trial was grouped into one of two categories: a sequential or interleaving category. If participants first entered the amount and account code of one expense before entering the other expense, this trial was grouped in the sequential category. If participants entered amounts of each expense first, and then entered account codes, or the other way around, this trial was grouped in the interleaving category. On a small subset of trials (<1\%) neither of these strategies was chosen: for example, participants first entered the amount of one expense, followed by the account code of the second expense. These trials were also grouped in the interleaving category, as participants switched to entering the second expense before completing the first expense.

Mouse clicks to switch between pages were used to code the order of people's actions, and get insight into the order in which people visited and entered data items. During the second iteration of grouping, for each trial the order of actions was considered and the trial was either grouped under a new strategy group for this order, or the trial was grouped under an existing strategy group. 

\subsection{Results}
Table \ref{tbl:ch34_4-means} summarises the results of the dependent measures for the three conditions. The distribution of dependent measures were skewed, so non-parametric Kruskal-Wallis tests were used to analyse effects of time costs on the dependent variables. A p-value of 0.05 was used for assessing the significance of all statistical tests. 

\begin{table}
 \includegraphics[width=\textwidth]{images/ch34/ch34-means.pdf}
\caption{The means (and standard deviations) of all dependent measures for each condition. The rates are calculated by dividing the number of occurrences to the number of opportunities, e.g. an interleaving rate of 50 percent means participants interleaved on 50 percent of trials.}
\label{tbl:ch34_4-means}
\end{table}

\subsubsection{Interleaving strategies}
A trial was labelled as 'interleaving' if the participant started entering one expense but interleaved to the other expense before completing the first one. The interleaving rate for each condition was calculated by dividing the number of trials where people interleaved by the number of total trials.  

The boxplots in Figure \ref{fig:ch34_4-boxplots} show the variability of interleaving rates across conditions. The Control condition had a median interleaving rate of 6\%, the High-Amount conditions had a median interleaving rate of 12\%, and the High-Account condition had a median interleaving rate of 96\%.

\begin{figure}
 \includegraphics[width=0.6\textwidth]{images/ch34/ch4_4-boxplot.pdf}
\caption{Boxplot of interleaving rates in each condition.}
\label{fig:ch34_4-boxplots}
\end{figure}

\begin{figure}
 \includegraphics[width=\textwidth]{images/ch34/ch34-4_linechart.pdf}
\caption{Line graph showing the frequency of interleaving rates for each condition. As can be seen, all three lines have two peaks at 0 and 100, which means that most participants interleaved on 0\% or 100\% of all trials.}
\label{fig:ch34_4-linechart}
\end{figure}

Participants interleaved most often between expenses in the High-Account condition (M = 73.2\%, SD = 41.1\%), compared to the Control (M = 31.17\%, SD = 42.24\%) and High Amount (M = 34.18\%, SD = 41.5\% ) conditions, $\chi^2$(2) = 6.81, p = 0.03. A post-hoc Dunn's test showed there was a difference between the High-Account condition and the Control (p = 0.02) and the High-Amount (p = 0.03) conditions, but not between the Control and High-Amount conditions (p=0.9).

Across conditions, most participants were consistent in their strategy choice, and either interleaved between expenses on almost no (0\%) or all (100\%) trials. This is illustrated in Figure \ref{fig:ch34_4-linechart}, which shows the distribution of interleaving rates for each condition. The lines all have peaks at the left and right end, indicating the interleaving rate was predominantly 0 or 100\% in each condition. Graphs of each individual participant are included in Appendix \ref{ch:S4_PartPlot}, which shows per trial whether a participant interleaved or not. These graphs further illustrate that participants used the same strategy throughout the experiment.

\subsubsection{Number and duration of visits}
There was no difference in the number of visits, $\chi^2$(2) = 2.90, p = 0.23. On average, participants made 4 visits per trial (i.e. one visit per data entry). Participants visited an information page for 1.8 seconds on average, and there was no significant difference in duration of visits between conditions, $\chi^2$(2) = 0.30 p= 0.8. 

\subsubsection{Most common order of inquiries}
To get a better insight in the specific order in which participants viewed and entered items, the trials were grouped based on the order of actions. There were eight different possible actions: viewing the first amount (V-Am1), viewing the second amount (V-Am2), viewing the first account code (V-Acc1), viewing the second account code (V-Acc2), entering the first amount (E-Am1), entering the second amount (E-Am2), entering the first account code (E-Acc1), and entering the second account code (E-Acc2). This iteration of grouping the trials resulted in 17 different strategy groups in total, with the majority of trials (98\%) grouped in the same four groups, which are shown in Figure \ref{fig:ch34_4-groupstr}. For example, the first strategy (a) shows a strategy where participants started a trial by visiting the Amount page, and then visiting the Accounts page. They then entered both the amounts of the first expense (Am1) and the account code of the first expense (Acc1). They then visited the Amounts page again, and entered the amount of the second expense (Am2), and then visited the Accounts page again and entered the account code of the second expense (Acc2). Table \ref{tbl:ch34_4-groupstr} shows the frequency with which these strategies were chosen per condition.

In the High-Account condition, participants predominantly switched to the page with the Amounts first, which had no delay, and entered these into the data entry form. In the other two conditions, participants mostly entered an amount and account code of the first expense first, and then entered the amount and account code of the second row. Figure x shows the most common variations of the order in which data was inquired and entered.


\begin{figure}[!ht]
  \centering
    \includegraphics[width=0.5\textwidth]{images/ch34/ch34_4-groupstr.png}
      \caption{The sequence of the most common grouping strategies. V = visit to the data source, E = entry of the data item. For example, in Strategy (a) a participant first viewed Amount1 and Account1, and then entered Amount1 and Account1. He/she then viewed Amount2 and entered it, and then viewed Account2 and entered it.}
          \label{fig:ch34_4-groupstr}
\end{figure}

\begin{table}[!ht]
\centering
\resizebox{\textwidth}{!}{
\begin{tabular}{l|l|l|l|l|ll}
\cline{2-5}
                                   & \multicolumn{2}{l|}{Sequential}                                                      & \multicolumn{2}{l|}{Interleaving}                                                    &                                  &                                \\ \hline
\multicolumn{1}{|l|}{Condition}    & First row (a)     & \begin{tabular}[c]{@{}l@{}}First \&\\ Second row (b) \end{tabular} & Amounts (c)       & \begin{tabular}[c]{@{}l@{}}Amounts \& \\ Accounts (d) \end{tabular} & \multicolumn{1}{l|}{Other}       & \multicolumn{1}{l|}{Total}     \\ \hline
\multicolumn{1}{|l|}{High-Account} & 34\%  {\footnotesize (48)}     & 4\%  {\footnotesize (6)}                                                       & 57\%  {\footnotesize (80)}     & 2\%  {\footnotesize (3)}                                                        & \multicolumn{1}{l|}{3\%  {\footnotesize (4)}}     & \multicolumn{1}{l|}{100  {\footnotesize (141)}} \\ \hline
\multicolumn{1}{|l|}{High-Amount}  & 20.99\%  {\footnotesize (44)}  & 16.57\%  {\footnotesize (35)}                                                  & 49.72\%  {\footnotesize (104)} & 9.39\% {\footnotesize  (20)}                                                    & \multicolumn{1}{l|}{3.31\%  {\footnotesize (7)}}  & \multicolumn{1}{l|}{100  {\footnotesize (210)}} \\ \hline
\multicolumn{1}{|l|}{Control}      & 11.2\%  {\footnotesize (16)}   & 21.6\%  {\footnotesize (32)}                                                   & 54.4  {\footnotesize (81)}     & 12\%  {\footnotesize (18) }                                                     & \multicolumn{1}{l|}{0.8\%  {\footnotesize (1)}}     & \multicolumn{1}{l|}{100  {\footnotesize (148)}} \\ \hline
\multicolumn{1}{|l|}{Total}        & 21.18\%  {\footnotesize (190)} & 15.02\%  {\footnotesize (73)}                                                  & 52.96\%  {\footnotesize (265)} & 8.37\%  {\footnotesize (41)}                                                    & \multicolumn{1}{l|}{2.46\% (12)} & \multicolumn{1}{l|}{100  {\footnotesize (499)}} \\ \hline
\end{tabular}
}
\caption{The most common grouping strategy was to chunk the items into three groups. The strategies are shown graphically in Figure \ref{fig:ch34_4-groupstr}.}\label{tbl:ch34_4-groupstr}
\end{table}

\subsubsection{Task performance}
There were 200 data entries, so in total there were 200 opportunities for a participant to make a data entry error. The error rates were calculated as the number of errors divided by the number of entries. Though the mean error rate was higher in the Control condition (M=8.68\%, SD=10.90\%) compared to the High-Amount (M=3.77\%, SD=2.79\%) and High-Account (M=5.18\%, SD=4.13\%) conditions, this difference was not statistically significant, $\chi^2$(2) = 0.41, p = 0.8. 
The High-Cost conditions had an extra time cost to overall completion time, due to the delay to one of the windows. Therefore, two completion times were calculated: one measure considered the actual completion time with the delay times included, and another measure considered the completion time with the delay times removed. Considering these two times, there was no difference in the time it took to complete a trial using the actual completion time,  X(2) = 0.15, p= 0.9, and with the delay times removed,  $\chi^2$(2) = 2.92, p = 0.2. On average, participants took about 29 seconds per trial across conditions.

\citet{Wiseman2011} taxonomy of number entry errors was used to analyse the types of data entry errors that were made. As can be seen in Figure x , the most prominent error types were when participants had a digit(s) wrong (60 times), when a data entry was skipped (75 times) or when they entered a correct number, but in the wrong input field (57 times): these types of errors make up for 61\% of all errors.

\subsubsection{Qualitative findings}
After the experiment had ended, participants were debriefed and the purpose of the study was explained. Some participants reflected on their strategies and gave additional explanations behind them. While these explanations are not the main focus of analysis and only serve to complement the quantitative measures, it helps understand people's motivation behind some of the measured strategies.

Participants mentioned they adapted their strategy several times throughout the experiment, in order to find the quickest way to complete the task. Because amounts were shorter and easier to remember, five participants mentioned they tried to first view all amounts before entering them. They tried this strategy with account codes as well, but these were longer and therefore it was more difficult to memorise two items at a time. As a result, most participants ended up viewing and entering each account code one by one. This type of strategy is illustrated in Figure \ref{fig:ch34_4-groupstr}.

Four participants noticed that numbers re-occurred throughout the experiment. They felt it was easier to memorise a number that had already occurred earlier in the experiment, so when a trial contained a number they recognised, they would memorise this item as well as another item, before returning to the entry form. If they did not recognise the number, they would memorise one item. Furthermore, as data items had a fixed length, some participants started a trial by entering placeholders: they entered amounts of four digits and a decimal point, and account codes of six digits. They would then visit the information pages to check which of the digits of the items they needed to change. 

\subsection{Discussion}
%Summary
The aim of this study was to understand the effect of time costs on the timing of inquiries. If there were no differences in time costs, participants completed a data entry sheet in sequential order, and completed one expense before moving to the next one. In the High-Cost conditions, people interleaved significantly more between expenses in the High-Account but not High-Amount condition. These findings partly support the hypothesis that people postpone inquiries with a high time cost, but it does not explain why participants entered the data entry sheet in sequential order in the High-Amount condition. 

\subsubsection{Timing of inquiries}
These results can be explained when considering the order in which the data was presented, and the order in which items were entered. Across conditions, participants predominantly started each trial by entering the first cell of the data entry sheet, the amount of the first expense, regardless of whether the Amounts window had a 2-s delay. However, the second item they entered was dependent upon which window had a delay: if the amounts window had a delay, participants would enter an account code first. If there was a delay with the accounts, they would enter the second amount first.
This behaviour suggests that time costs do not influence the first visit, but do affect subsequent visits. Even though the time cost was consistent throughout the experiment, potentially the experiment was too short for participants to learn which of the windows had a delay and only adapted their strategy after they had already entered the first item. Furthermore, participants tended to stick to the same strategy they had started with throughout the experiment.

The finding that participants postpone inquiries with a high time cost is in line with findings from Study 2 and suggests people schedule their inquiries more efficiently and effectively. Though there was no measured difference in task performance in the study, long interruptions have been shown to be disruptive \citep{Altmann2017, Monk2008}, and leaving these until a natural breakpoint can reduce errors, as it is easier to resume a task \citep{Gould2013a, Iqbal2005}.

\subsubsection{Chunking of data items}
In contrast with Study 3 and prior work \citep{Gray2006}, an increase in time costs did not reduce the number of visits. However, in these prior studies there was no interaction involved to view information in the Low-Cost condition: information was permanently visible in the task interface. In the current study, participants always had to move their mouse and click in order to view the information pages, which may have encouraged them to try and reduce visits and chunk items even in the Control condition. The time cost affected which items participants chunked together, but not whether they chunked items or not.

\subsubsection{Transfer of strategies}
People adapted their strategies even if only some, but not all, of the information was hard to access. Exposure to time costs may have made people adapt their strategies for all inquiries. This transfer of strategies is consistent with previous research, that has shown a more memory-based strategy can be trained and transferred to other situations where the cost to access information is no longer high \citep{Patrick2014}. This study extends these findings by showing that inquiry strategies can also transfer within a task, when the user has to access multiple information sources with both a low and high time cost.

\subsubsection{Conclusion}
We learn from this study that people avoid time costs by postponing inquiries with an increased time cost. In this study, both expenses were shown on the same window, and could be seen as part of the same task. Workers in Study 2 however not only dealt with multiple information windows, but also multiple data entry windows, and had to be careful not to enter information in the wrong windows. What we do not know from Study 4 is whether participants will also avoid time costs of inquiries by switching between data entry tasks. Based on the results of the current study, the hypothesis is made that a difference in time costs makes people more likely to interleave between different data entry tasks to enter items with a low time cost first. This hypothesis is tested in Study 5.

\section{Study 5: Interleaving between data entry tasks}
 
\subsection{Introduction}
Study 3 showed that people avoid time costs by reducing the number of inquiries. Study 4 suggested that people avoid time costs by postponing them. In these studies, people were only presented with one task at a time, while setting in Study 2 coordinate multiple tasks. How would people deal with time costs when they have two tasks? Participants in Study 1 and 2 avoided switches to tasks that were completely unrelated to their data entry work, but people may switch between similar data entry tasks if it makes them faster. For instance, upon opening a spreadsheet that took time to retrieve, it may be more efficient to enter the account codes from that spreadsheet for multiple tasks. However, multitasking can be prone to errors. 

Prior research, studying the effect of time costs on multitasking in a hospital setting, found that increased time costs reduces multitasking. \citet{Back2012} conducted a lab experiment where participants had to enter information from a prescription form into two simulated infusion pumps. For each pump, they had to enter two types of information: the medication dose and the time duration. If the form was physically further away from the pumps, participants more often completed one pump before starting another and as a result made fewer errors in omitting a task step. The higher access cost had the effect that participants memorised and chunked information on the form according to the pump rather than type of information, which reduced multitasking.

However, in Back et al.'s study, all information was located on one information source, and participants incurred a single cost to access it. People therefore chunked information to memorise as much information per visit as possible, so that they did not have to revisit the source too often. It is unclear what the effect of time costs is in scenarios where people do not have to get multiple information items from one source, but rather information from multiple sources. How do people prioritise which information to look up first? Do they still complete looking up information for one task first, before starting another task?

The aim of Study 5 is to test whether effect of time costs, as found in Study 4, extend to a multi-task paradigm. Participants were asked to complete an experiment similar to the task in Study 4, but had to complete two tasks per trial.

\subsection{Method}
\subsubsection{Participants}
Fourty-two participants (32 female), ranging from 18-46 years (M = 25, SD= 8) took part in the experiment. They were recruited from a university subject pool and received $\pounds$4 for their participation.

\subsubsection{Materials}
The task was similar to the one used in Study 4 but differed in one aspect. Instead of filling in one data entry form per trial, participants had to complete two forms per trial, which were shown on two different pages (see Figure \ref{fig:ch34_5-tasklayout}.Each data entry sheet contained one expense, and participants completed the trial by entering the amount and account code for each sheet. The aim of this follow-up study was to investigate if differences in IAC of the two sources makes people more likely to interleave between two separate data entry tasks.

\begin{figure}
\includegraphics[width=\textwidth]{images/ch34/ch34-5_Tasksequence.pdf}
    \caption{Participants had to enter two data entry forms per trial, each containing two items. Each trial started by showing the first data entry form. As in Study 4, the data items for both forms were retrieved from a separate Amounts page (Step 1) and an Accounts page (Step 2). Participants had to enter the items for the first form (Step 3) and second form (Step 4) before submitting the data entries and moving on to the next trial.}\label{fig:ch34_5-tasklayout}
\end{figure}

\subsubsection{Design}

The experiment was a between-participants design with the presence of a delay as the independent variable. As in Study 4, in the Control condition there were no delays in opening the pages. In the High-Amount condition, there was a delay in opening the page with amounts. In the High-Account condition, there was a delay in opening the page with account codes. The main dependent variable was whether participants interleaved between sheets or not: did participants enter the data items in sequential order, or did they interleave between the two sheets? If participants entered the amount and account code of one sheet before entering the other sheet, this was considered a sequential order. If participants entered amounts of each sheet first, followed by entering the account codes or vice versa, this was considered interleaving. All key presses were recorded to determine in which order data was entered. Page switches were recorded to capture when and how often a participant looked up the data items. Other dependent variables were trial completion time, data entry error rate, and type of errors.

\subsubsection{Procedure}
The experimental setup was similar to Study 4. For each experimental trial, participants had to enter four data items: they had to complete two forms with two entries each, an account code and an amount. For each experimental trial, participants had to enter four data items, two for each sheet. It was explained that they could use any strategy they wanted, but that it was important to complete both sheets before continuing to the next trial. Participants first completed two practice trials to familiarise themselves with the task, and data from the practice trials were excluded from the analysis. The experiment took approximately 30 minutes.

\subsection{Results}
Table \ref{tbl:ch34_5-means} shows a summary of the results of all three conditions for the dependent variables. Kruskal-Wallis tests were carried out to test if there were significant differences between the conditions.

\subsubsection{Cleaning up the data}
Three participants were removed from the data due to extreme values on performance measures.
P28 and P23 made at least one error on every trial. They made 118 and 153 errors out of 200 error opportunities, respectively. P26's session was terminated before the end had been reached, as 45 minutes had passed. This participant spent on average 65 seconds per trial, which is twice as long as the mean trial time of other participants.

These three participants were considered outliers and removed from the data. Data of the remaining 39 participants was taken into the data analysis.

\begin{table}
 \includegraphics[width=\textwidth]{images/ch34/ch34_5-means.pdf}
\caption{The means (and standard deviations) of all dependent measures for each condition. The rates are calculated by dividing the number of occurrences to the number of opportunities, e.g. an interleaving rate of 50 percent means participants interleaved on 50 percent of trials.}
\label{tbl:ch34_5-means}
\end{table}

\subsubsection{Interleaving strategies}
A trial was labelled as 'interleaving' if the participant started entering one data entry sheet, but interleaved to entering items on the other sheet before completing the first one. The interleaving rate for each condition was calculated by dividing the number of trials where people interleaved by the number of total trials. 

The boxplots in Figure 4 show the variability of interleaving rates across conditions. Participants interleaved most often between data entry sheets in the High-Account (M = 73.4\%, SD = 32.1\%) and High Amount (M = 83.8\%, SD = 21.6\% ) conditions compared to the Control (M = 30.5\%, SD = 37.7\%) condition, $\chi^2$(2) = 11.13, p = 0.004. A post-hoc comparison showed there was a differnece between the Control and the High-Amount (p<.01) and High-Account (p = 0.01) conditions, and no difference between the High-Account condition and the High-Amount (W = 22, p = 0.4) conditions.

As can be seen in Figure \ref{fig:ch34_5-linechart}, which shows the distribution of interleaving rates, all participants in the High-Cost conditions interleaved on at least a part of the trials. This is illustrated by the left side of the graph: the lines of the High-Cost conditions have a frequency of 0 participants at an interleaving rate of 0\%. The Control condition line has no obvious peak, indicating that interleaving rates in this condition were evenly distributed: participants interleaved on zero, a portion, as well as all of the trials.

As in Study 1, participants made on average four visits per trial, i.e. one visit per data entry. There was no difference in the number of visits, $\chi^2$(2) = 1.59, p = 0.5. Participants made significantly shorter visits in the Control (M = 2.00s, SD = 0.68s) condition compared to the High-Account condition (M = 2.25s, SD = 0.67s) compared to the High-Amount (M = 2.61s, SD = 0.85s) and  $\chi^2$(2) = 6.14, p= 0.04. Post-hoc comparisons found a significant difference between  the High-Amount and the Control (p=.02) conditions, but not between High-Account and Control conditions (p = 0.2) or the High-Account and the High-Amount (p = 0.2).

\begin{figure}
 \includegraphics[width=\textwidth]{images/ch34/ch34-5_linechart.pdf}
\caption{Line graph showing the frequency of interleaving rates for each condition. It can be seen that in the Control condition, the line is even, which means that an even distribution of participants interleaved on all, a portion, or all trials. The lines of the High-Cost conditions peak at the right end, which means most participants in these conditions interleaved on at least a portion if not 100\% of all trials.}
\label{fig:ch34_5-linechart}
\end{figure}

\subsubsection{Errors and trial completion time}
There were 200 data entries, so in total there were 200 opportunities for a participant to make a data entry error. The error rates were calculated as the number of errors divided by the number of entries. 
There was a marginal though not significant effect of time cost on error rate, $\chi^2$(2) = 5.37, p = 0.06. The mean error rate was marginally higher in the High-Account condition (M= 8.42\%, SD=9.08\%) compared with the High-Amount (M=7.54\%, SD=4.33\%) and Control (M=3.88\%, SD=4.13\%) conditions. When comparing the actual completion time including lockouts, participants were significantly faster in the Control condition (M=27.39, SD = 3.49s) than the High-Account (M = 33.83s, SD = 6.08s) or High-Amount (M = 33.11s, SD = 8.16s) conditions,  $\chi^2$(2) = 8.52, p= 0.01. With the lockout times removed, the difference is no longer significant, $\chi^2$(2) = 1.61, p = 0.4.

The type of errors can be seen in Figure. The most common error type was when a data entry was skipped: this happened 243 times. Table 1 shows the number of skipped errors for each condition. It can be seen that in the Control condition this type of error occurred 16 times. The error happened more frequently in the High-Cost conditions: in the High-Account condition it happened 114 times, and in the High-Amount condition it happened 116 times.
Typing the correct number but in the wrong field happened 78 times. This happened 18 times in the Control condition, 14 times in the High-Account and 46 times in the High-Amount condition.
When comparing across conditions, these types of errors happened on a significantly higher proportion of data entries in the High-Account (M = 4.58\%, SD = 3.6\%) and High-Amount (M=6.54\%, SD=5.01\%) compared with the Control condition (M = 1.23\%, SD = 1.82\%),  $\chi^2$(2) = 11.29, p = 0.004.  A post-hoc comparison showed there was a difference between the Control and the High-Amount (p<.01) and High-Account (p = 0.01) conditions, and no difference between the High-Account condition and the High-Amount (W = 22, p = 0.4) conditions.

\subsection{Discussion}
%Effect of time costs
The results of Study 5 show that participants in High-Cost conditions avoided time costs by entering items with a low cost first, which meant they interleaved between expenses and entered items of each expense that had a low cost first. Though interleaving also happened in the Control condition, it happened significantly more often in the High-Cost conditions. These results further support findings from Study 3 and 4 that people avoid time costs and try to minimise time, and the results show that this effect extends beyond a single task setup.

The findings are consistent with the soft constraints hypothesis that people are sensitive, and adapt their strategies, to milliseconds \citep{Charman 2003, Gray2000}. In previous studies, it was shown how time costs affect the number of steps taken to complete a task \citep{Gray2006}. Study 4 and 5 contribute to this line of work by showing time costs also affect the order of steps in a routine task.

%Interleaving
The finding that interleaved more as time costs increased, contrasts with \citet{Back2012}, who found that an increase in time costs made people less likely to interleave between two data entry tasks. This may be due to the presentation of the information. In \citet{Back2012}'s study, people had to retrieve all information for both data entry tasks from one sheet. If the sheet was nearby, participants read one item at a time, and interleaved between tasks on 59\% of the trials.  As the cost to access this source increased, they chunked the data items associated with one task, and then after completing this task, returned to the source to chunk data items for the second task. 

Contrary to prior research, there was no difference in number or duration of visits. Across conditions, participants made one switch per data entry which is probably the maximum amount people can reliably hold in short-term memory. 

%Contribution and implication
This study contributes to our understanding of how time costs affect task switching behaviour, and can have implications for tools aimed to minimise task switches. In the current study, there was only a time cost when switching to one of the information sources, but not when switching between tasks. The results showed that people try to avoid switching to something with a high cost. Therefore, adding a cost when switching between tasks may encourage people to complete one task, before switching to another task.

Furthermore, even though the two data entry tasks were separated on different windows, participants may still have felt it was part of the same activity, as the two data entry tasks were shown in the same interface. Study 2 suggests that even though people tried to avoid task switches, switches that were seen as part of the activity were commonly observed. 

\section{Summary}
The aim of this chapter was to investigate the effect of time costs on the number, duration and timing of inquiries for a data entry task. Study 3 showed that if people retrieve all data from the same source, they will reduce switches between entering and looking up data if the access costs to this source increases. As it took more time to access, offloading behaviour was observed as well, and several participants prepared items they were going to need nearby, but did not use them yet. Study 4 further demonstrates that when people have to retrieve data from multiple sources, they collect and group items that are quick to access first, and leave items that take longer to access until the end. Study 5 demonstrated the robustness of the effect of time costs in a multi-task setup: when dealing with two data entry tasks, people still entered items with a low time cost first, and interleaved between tasks to enter items with low costs first. As a result, participants made more omission errors and submitted tasks before they had completed entering all the items. These studies contribute to our understanding of time costs on self-interruption behaviour to collect information: if people know the expected time duration of an interruption, they make fewer interruptions that are long and postpone these switches. 

Office workers in Study 2 took time costs into account in a similar manner when managing physical interruptions, but did not adopt the same strategy for digital interruptions: these were addressed immediately as participants presumed these to be quick. Comparing those results with the results reported in this chapter, it suggests that people may not be aware of time costs of digital interruptions in a naturalistic setting. In the experiments in this chapter, time costs were manipulated in a specific way: a time delay was added to the interface to reveal information. In practice, the time spent on an inquiry may be because of the time to access it, but also because of time spent searching for information, or people get distracted, and further self-interrupt to other activities. These factors may further make it difficult for participants to learn the time costs associated with interruptions and adapt their self-interruption behaviour. 

The studies so far have shown that people try to minimise time to complete their data entry work in an efficient way, but are not aware of the time they spend away from their task looking up information required for the task. The next chapter explores whether a design intervention showing people how long they go away for can make people more aware of these digital interruptions, and whether this has an effect on interruption strategies and task performance.