\chapter{Evaluating design changes to improve performance in data entry with varying IACs}

\begin{mynote}
\subsubsection{Chapter outline}
The previous chapter demonstrated that IAC affects how people switch between entering and looking up information. This chapter describes two studies that, given varying IACs, explore the extent to which design changes to the data entry interface can influence strategies, speed and accuracy. Study 5 evaluates these in a controlled setting, to see if making design changes influence the strategies people adopt, and can make people adopt more accurate and/or efficient strategies. Study 6 evaluates them in the office setting, to ascertain how appropriate the proposed recommendations are in the context in which they would be used.

Together these studies intend to show that the new interface makes people switch less between entering and looking up information, which makes them faster to complete the data entry task overall and can reduce errors.
\end{mynote}

\section{Introduction}

The findings of the previous studies will have given insight into the influence of IAC on people's strategies of managing looking up information and how different strategies may be more efficient and accurate than others. For example, one finding can be that people who look up information from high IAC sources as they need it are slower and make more data entry errors than people who first collect all information and then enter it all in one sequence. 
It will have highlighted some functionalities that a data entry expenses system needs to offer users. These findings are translated into a set of requirements. These are used to test the existing system against, and used to develop possible future design recommendations suited to the task of entering expenses. 

The design recommendations will take into account both findings from Study 3 and 4 on what influences people's strategies and what is desirable, as well as the setting studied in Study 1 and 2 and what is feasible. For example, desired changes in the actual interface may be too expensive to be realistic, and it may be more feasible to change the way information sources are designed, or how these are laid out in the user's environment. Screenshots of the current interface system will be used (initial ones were obtained in Study 1, additional ones will be obtained in Study 2). 

Study 5 aims to test different designs in a controlled experiment, to investigate if changing design features influences people's switching strategies and their speed and accuracy in data entry. It will use the same task paradigm as Study 4 and compare different designs, to see if these changes have an influence on the strategies people adopt in looking up information for a data entry task, and whether these changes can make people adopt strategies that improve accuracy. 

Study 6 aims to evaluate the design recommendations in a finance office with workers, to see how appropriate and feasible the proposed recommendations would be in the context for which they are developed. Depending on what the design recommendations will be, I will take them through a prototype, which can be a paper prototype, storyboard, or digital mockup, and if possible I will ask them to perform the task with and without the proposed changes. 

\subsection{Contributions}
\begin{itemize}
\item
Development of design recommendations for an expenses system.
\item
Demonstrate how an understanding of the used information sources and people's switching strategies between entering and looking up information can be used to adapt the design of the data entry interface. 
\item
Demonstrate the applicability of design recommendations in the financial office settings in which the expenses task is currently done. 
\item
Demonstrate that design features can influence people's strategies in entering expenses in a financial office setting.
\end{itemize}

