\section*{Impact statement}
\addcontentsline{toc}{chapter}{Impact statement}

The work presented in this thesis has impacts both inside and outside academia. 

Within academia, this thesis has increased our understanding of how time costs affect how people manage self-interruptions to look up information. These interruptions are addressed straight away if they are presumed to be quick and it is considered part of the activity, but it is hard for people to predict time costs outside of a controlled setting. Interruptions can take much longer than expected, which increases their disruptiveness to work. My thesis demonstrates that showing people how long they go away from a specific task helps people reduce the length of their interruptions, which can inform future work on understanding and controlling interruption behaviour. The findings have been published and presented at HCI conferences, universities and research labs. 

The findings also make an impact to the methodology of future data entry research, by demonstrating that for some types of data entry work, a major component is collecting data which impacts how data is entered. If data entry interfaces are to be used in situations where information is not readily available, they should be evaluated  by requiring participants to collect data from the environment. As part of this thesis I developed a new experimental task, which can be used in future data entry studies to investigate time costs of collecting data during a data entry task.

Outside academia, the findings of this thesis have a direct impact on developers of interruption management tools. This thesis contributes to our understanding of how awareness of time spent on technology can help people manage their use of time. Time management of technology use is a current issue: various technology companies have started to introduce features in 2018 to show how much time is spent in their applications, to encourage users to adopt healthier technology habits. My work contributes to addressing this issue by demonstrating how time awareness can not only help to regulate work-irrelevant technology use, but also to manage task-relevant interruptions during work. As part of my thesis, I have developed and evaluated a browser extension showing how much time the user spends away from a task: the extension is publicly available online to download and use. 